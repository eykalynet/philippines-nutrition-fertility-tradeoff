% AER-Article.tex for AEA last revised 22 June 2011
\documentclass[]{AEA}

% The mathtime package uses a Times font instead of Computer Modern.
% Uncomment the line below if you wish to use the mathtime package:
%\usepackage[cmbold]{mathtime}
% Note that miktex, by default, configures the mathtime package to use commercial fonts
% which you may not have. If you would like to use mathtime but you are seeing error
% messages about missing fonts (mtex.pfb, mtsy.pfb, or rmtmi.pfb) then please see
% the technical support document at http://www.aeaweb.org/templates/technical_support.pdf
% for instructions on fixing this problem.

% Note: you may use either harvard or natbib (but not both) to provide a wider
% variety of citation commands than latex supports natively. See below.

% Uncomment the next line to use the natbib package with bibtex
\usepackage{natbib}

% Uncomment the next line to use the harvard package with bibtex
%\usepackage[abbr]{harvard}

% This command determines the leading (vertical space between lines) in draft mode
% with 1.5 corresponding to "double" spacing.
\draftSpacing{1.5}


% tightlist command for lists without linebreak
\providecommand{\tightlist}{%
  \setlength{\itemsep}{0pt}\setlength{\parskip}{0pt}}


% Add imagehandling




\usepackage{hyperref}

\begin{document}


\title{Nutritional Deficits and the Quantity-Quality Trade-off: Evidence
from an Exogenous Fertility Shock in Low-Income Urban Settings in the
Philippines}
\shortTitle{The Beckerian Quantity-Quality Trade-off in the Philippine
Context}
% \author{Author1 and Author2\thanks{Surname1: affiliation1, address1, email1.
% Surname2: affiliation2, address2, email2. Acknowledgements}}


\author{
  Erika Salvador\\
  Caroline Theoharides\thanks{
  Salvador: Amherst
College, \href{mailto:esalvador28@amherst.edu}{esalvador28@amherst.edu}.
  Theoharides: Amherst
College, \href{mailto:ctheoharides@amherst.edu}{ctheoharides@amherst.edu}.
  This research was made possible by the Schupf Fellowship at Amherst
  College. I am indebted to my faculty adviser, Professor Caroline
  Theoharides, for her mentorship and support throughout the development
  of this project. I also wish to thank Faculty Director Professor
  Amelie Hastie and the campus partners whose institutional support
  enabled this work. I am further grateful to the Departments of
  Economics and Mathematics \& Statistics for their academic support.
  The views expressed and any errors contained herein are entirely my
  own responsibility.
}
}

\date{\today}
\pubMonth{06}
\pubYear{2025}
\pubVolume{1}
\pubIssue{1}
\JEL{J13, I15, O15}
\Keywords{fertility shocks, child nutrition, quantity-quality
trade-off, urban poverty, Philippines}

\begin{abstract}
This paper examines whether increased fertility affects early-life
nutritional outcomes in low-income urban households. I exploit a natural
experiment created by a 1990 policy in Manila, Philippines, which banned
modern contraceptives from city-run health facilities. Using a
difference-in-differences framework and nationally representative data
from the Philippine Demographic and Health Surveys, I estimate the
reduced-form impact of the policy on child height-for-age and
weight-for-height. {[}Will add more after data analysis{]}
\end{abstract}


\maketitle

\section{Introduction}

The trade-off between child quantity and child quality is a foundational
concept in the economics of the family. First articulated by
\citet{becker1960economic} and extended in subsequent models of
household behavior \citep{beckerlewis1973, beckertomes1976}, this
framework posits that parents allocate finite resources---both financial
and non-financial---across children. An increase in fertility reduces
the resources available per child and, under binding constraints, may
lead to lower investments in health, education, and other forms of human
capital. This mechanism has served as an explanatory model for changes
in fertility behavior and the evolution of population structures in low-
and middle-income countries.

Empirical investigations of the quantity--quality trade-off have focused
primarily on educational outcomes. Studies in both high-income and
low-income settings have examined the effects of fertility on school
enrollment, grade progression, test scores, and completed years of
schooling
\citep{rosenzweig1980testing, black2005more, angrist2010effects}. These
outcomes serve as accessible proxies for long-run human capital
accumulation, but they represent only one dimension of child quality.
Other outcomes, such as nutritional status, are early-onset or
biologically constrained. They also tend to be less responsive to
remediation later in life. A child who suffers from chronic malnutrition
may exhibit permanently reduced cognitive capacity and face limits in
physical development that affect long-run productivity regardless of
subsequent educational access
\citep{hoddinott2013adult, grantham2007development}.

The exclusion of nutritional outcomes from much of the empirical
literature leaves an important dimension of the trade-off untested.
Nutritional investments in early childhood are essential to early
childhood development and long-term outcomes
\citep{victora2008maternal, hoddinott2013adult}. They shape brain
development, immune system functioning, and physical stature, and they
have been shown to predict later-life earnings and health outcomes
across a wide range of settings
\citep{grantham2007development, alderman2006long}. The biological
irreversibility of early-life nutritional deficits further distinguishes
them from other forms of investment. Educational deficits may be
partially remediable; nutritional failures often are not. A credible
estimate of the trade-off between fertility and child quality must
account for nutrition if it aims to assess the full set of consequences
associated with fertility shocks.

This study addresses this gap by examining the nutritional effects of a
localized, exogenous increase in fertility in the Philippines. In 1990,
the mayor of Manila implemented an executive policy that prohibited the
provision of modern contraceptives in all city-run health facilities.
The order removed access to pills, condoms, intrauterine devices, and
related public health materials and instructed healthcare providers to
offer only natural family planning methods. This policy remained in
place for nearly a decade and affected only the jurisdiction of the
Manila city government. The national government did not implement a
comparable restriction, and surrounding cities within Metro Manila
continued to provide access to modern contraceptives. The policy thus
created a spatial and temporal discontinuity in contraceptive access
that was uncorrelated with underlying fertility preferences or
concurrent shifts in household income or governance. As a result, the
Manila ban serves as a quasi-experimental source of variation in
fertility exposure among poor urban households.

I use this natural experiment to estimate the causal effect of increased
fertility on child nutrition. The analysis relies on nationally
representative data from multiple waves of the Philippine Demographic
and Health Survey (DHS), which provide data on household structure and
maternal characteristics, as well as measurements of child
anthropometry. The outcomes of interest are height-for-age and
weight-for-height z-scores, which serve as standardized indicators of
chronic and acute malnutrition, respectively. These outcomes are widely
used in the global health and development literature and capture
nutritional deprivation over both long and short time horizons
\citep{victora2008maternal}. The empirical strategy follows a
difference-in-differences design that compares child outcomes in Manila
and comparable urban areas before and after the onset of the policy.

The identification strategy rests on two key assumptions. First, in the
absence of the contraceptive ban, nutritional trends in Manila would
have evolved in parallel with those in comparison cities. Second, any
other policy or economic shocks affecting Manila during the study period
must not coincide precisely with the timing and scope of the
contraceptive policy. I test these assumptions using falsification
checks, placebo comparisons, and robustness specifications that include
city-specific time trends, maternal fixed effects, and controls for
baseline demographic differences.

The analysis proceeds in three stages. I first replicate existing work
\citep{dumas2010fertility} to confirm that the contraceptive ban led to
an increase in fertility among affected women. I then estimate
reduced-form effects of policy exposure on nutritional outcomes for
children under five years of age. Finally, I examine heterogeneity in
effects across subsamples defined by maternal education, household
wealth, and access to prenatal care. These dimensions serve as proxies
for household resource availability and capacity to buffer the
nutritional consequences of fertility increases.

This research contributes to the literature in several respects. It
introduces new evidence on the nutritional consequences of fertility
shocks in an urban developing-country context. It expands the
measurement of child quality beyond educational outcomes to include
biological indicators. It also adds to the limited body of work that
uses plausibly exogenous variation in reproductive health policy to
study downstream effects on human capital. More broadly, the findings
underscore the potential for local family planning restrictions to
generate unintended consequences for child well-being in settings
characterized by poverty, food insecurity, and institutional fragility.

\bibliographystyle{aea}
\bibliography{references}

\% The appendix command is issued once, prior to all appendices, if any.
\appendix

\section{Mathematical Appendix}


\end{document}
