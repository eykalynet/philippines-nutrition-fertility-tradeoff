% AER-Article.tex for AEA last revised 22 June 2011
\documentclass[]{AEA}

% The mathtime package uses a Times font instead of Computer Modern.
% Uncomment the line below if you wish to use the mathtime package:
%\usepackage[cmbold]{mathtime}
% Note that miktex, by default, configures the mathtime package to use commercial fonts
% which you may not have. If you would like to use mathtime but you are seeing error
% messages about missing fonts (mtex.pfb, mtsy.pfb, or rmtmi.pfb) then please see
% the technical support document at http://www.aeaweb.org/templates/technical_support.pdf
% for instructions on fixing this problem.

% Note: you may use either harvard or natbib (but not both) to provide a wider
% variety of citation commands than latex supports natively. See below.

% Uncomment the next line to use the natbib package with bibtex
\usepackage{natbib}

% Uncomment the next line to use the harvard package with bibtex
%\usepackage[abbr]{harvard}

% This command determines the leading (vertical space between lines) in draft mode
% with 1.5 corresponding to "double" spacing.
\draftSpacing{1.5}


% tightlist command for lists without linebreak
\providecommand{\tightlist}{%
  \setlength{\itemsep}{0pt}\setlength{\parskip}{0pt}}


% Add imagehandling




\usepackage{hyperref}

\begin{document}


\title{Nutritional Deficits and the Quantity-Quality Trade-off: Evidence
from an Exogenous Fertility Shock in Low-Income Urban Settings in the
Philippines}
\shortTitle{The Q-Q Trade-off in the Philippine Context}
% \author{Author1 and Author2\thanks{Surname1: affiliation1, address1, email1.
% Surname2: affiliation2, address2, email2. Acknowledgements}}


\author{
  Erika Salvador\\
  Caroline Theoharides\thanks{
  Salvador: Amherst
College, \href{mailto:esalvador28@amherst.edu}{esalvador28@amherst.edu}.
  Theoharides: Amherst
College, \href{mailto:ctheoharides@amherst.edu}{ctheoharides@amherst.edu}.
  This research was made possible by the Schupf Fellowship at Amherst
  College. I am indebted to my faculty adviser, Professor Caroline
  Theoharides, for her mentorship and support throughout the development
  of this project. I also wish to thank Faculty Director Professor
  Amelie Hastie and the campus partners whose institutional support
  enabled this work. I am further grateful to the Departments of
  Economics and Mathematics \& Statistics for their academic support.
  The views expressed and any errors contained herein are entirely my
  own responsibility.
}
}

\date{\today}
\pubMonth{07}
\pubYear{2025}
\pubVolume{}
\pubIssue{}
\JEL{J13, I15, O15}
\Keywords{fertility shocks, child nutrition, quantity-quality
trade-off, urban poverty, Philippines}

\begin{abstract}
This paper examines whether increased fertility affects early-life
nutritional outcomes in low-income urban households. I exploit a natural
experiment created by a 1990 policy in Manila, Philippines, which banned
modern contraceptives from city-run health facilities. Using a
difference-in-differences framework and nationally representative data
from the Philippine Demographic and Health Surveys, I estimate the
reduced-form impact of the policy on child height-for-age and
weight-for-height. {[}Will add more after data analysis{]}
\end{abstract}


\maketitle

\section{Introduction}

The trade-off between child quantity and child quality is a foundational
concept in the economics of the family. First articulated by
\citet{becker1960economic} and extended in subsequent models of
household behavior \citep{becker1973interaction, becker1976child}, this
framework posits that parents allocate finite resources---both financial
and non-financial---across children. An increase in fertility reduces
the resources available per child and, under binding constraints, may
lead to lower investments in health, education, and other forms of human
capital. This mechanism has served as an explanatory model for changes
in fertility behavior and the evolution of population structures in low-
and middle-income countries.

Empirical investigations of the quantity--quality trade-off have focused
primarily on educational outcomes. Studies in both high-income and
low-income settings have examined the effects of fertility on school
enrollment, grade progression, test scores, and completed years of
schooling
\citep{rosenzweig1980testing, black2005more, angrist2010multiple}. These
outcomes serve as accessible proxies for long-run human capital
accumulation, but they represent only one dimension of child quality.
Other outcomes, such as nutritional status, are early-onset or
biologically constrained. They also tend to be less responsive to
remediation later in life. A child who suffers from chronic malnutrition
may exhibit permanently reduced cognitive capacity and face limits in
physical development that affect long-run productivity regardless of
subsequent educational access
\citep{hoddinott2013adult, grantham2007development}.

The exclusion of nutritional outcomes from much of the empirical
literature leaves an important dimension of the trade-off untested.
Nutritional investments in early childhood are essential to early
childhood development and long-term outcomes
\citep{victora2008maternal, hoddinott2013adult}. They shape brain
development, immune system functioning, and physical stature, and they
have been shown to predict later-life earnings and health outcomes
across a wide range of settings
\citep{grantham2007development, alderman2006long}. The biological
irreversibility of early-life nutritional deficits further distinguishes
them from other forms of investment. Educational deficits may be
partially remediable; nutritional failures often are not. A credible
estimate of the trade-off between fertility and child quality must
account for nutrition if it aims to assess the full set of consequences
associated with fertility shocks.

This study addresses this gap by examining the nutritional effects of a
localized, exogenous increase in fertility in the Philippines. In 1990,
the mayor of Manila implemented an executive policy that prohibited the
provision of modern contraceptives in all city-run health facilities.
The order removed access to pills, condoms, intrauterine devices, and
related public health materials and instructed healthcare providers to
offer only natural family planning methods. This policy remained in
place for nearly a decade and affected only the jurisdiction of the
Manila city government. The national government did not implement a
comparable restriction, and surrounding cities within Metro Manila
continued to provide access to modern contraceptives. The policy thus
created a spatial and temporal discontinuity in contraceptive access
that was uncorrelated with underlying fertility preferences or
concurrent shifts in household income or governance. As a result, the
Manila ban serves as a quasi-experimental source of variation in
fertility exposure among poor urban households.

I use this natural experiment to estimate the causal effect of increased
fertility on child nutrition. The analysis relies on nationally
representative data from multiple waves of the Philippine Demographic
and Health Survey (DHS), which provide data on household structure and
maternal characteristics, as well as measurements of child
anthropometry. The outcomes of interest are height-for-age and
weight-for-height z-scores, which serve as standardized indicators of
chronic and acute malnutrition, respectively. These outcomes are widely
used in the global health and development literature and capture
nutritional deprivation over both long and short time horizons
\citep{victora2008maternal}. The empirical strategy follows a
difference-in-differences design that compares child outcomes in Manila
and comparable urban areas before and after the onset of the policy.

The identification strategy rests on two key assumptions. First, in the
absence of the contraceptive ban, nutritional trends in Manila would
have evolved in parallel with those in comparison cities. Second, any
other policy or economic shocks affecting Manila during the study period
must not coincide precisely with the timing and scope of the
contraceptive policy. I test these assumptions using falsification
checks, placebo comparisons, and robustness specifications that include
city-specific time trends, maternal fixed effects, and controls for
baseline demographic differences.

The analysis proceeds in three stages. I first replicate existing work
\citep{dumas2019sex} to confirm that the contraceptive ban led to an
increase in fertility among affected women. I then estimate reduced-form
effects of policy exposure on nutritional outcomes for children under
five years of age. Finally, I examine heterogeneity in effects across
subsamples defined by maternal education, household wealth, and access
to prenatal care. These dimensions serve as proxies for household
resource availability and capacity to buffer the nutritional
consequences of fertility increases.

This study contributes to the literature in several important ways. It
provides new evidence on how increases in fertility---caused by policy
restrictions on family planning---can affect child nutrition in poor,
urban communities. Most past research has focused on education, but this
study expands the idea of child quality to include biological measures
such as stunting and wasting. It also adds to the small number of
studies that use unexpected changes in reproductive health policy to
examine long-term effects on children's well-being. More broadly, the
results show that local restrictions on family planning can
unintentionally harm children's health, especially in settings where
families already face poverty, food insecurity, and limited public
services.

\section{Review of Related Literature}

The quantity--quality (Q--Q) theory, a central idea in modern family
economics, holds that parents face a trade-off between the number of
children and the ``quality'' of investment---such as education or
health---they can provide to each. Quality in this context refers to the
human capital of each child: attributes like education, health, and
nutrition that enhance a child's future productivity and well-being. The
genesis of this idea traces back to Gary Becker's seminal work around
1960, which for the first time treated children as economic goods
subject to parental choice and budget constraints
\citep{becker1960economic}. Becker argued that as families become
wealthier, they may not simply want more children, but rather
better-raised children, much as a household might prefer a
higher-quality car or house over a greater quantity of them. This
proposition led to a formal theory in which increases in income or
changes in economic conditions cause parents to substitute child quality
for quantity, consistent with historical patterns of lower fertility and
higher educational attainment during economic development
\citep{galor2000population}.

In what follows, I review the theoretical foundations of the Q-Q model
and its evolution in the literature. I begin with the static models of
Becker \citep{becker1960economic} and Becker--Lewis
\citep{becker1973interaction}, which first formalized the trade-off
within a household utility maximization framework. We then examine
extensions to dynamic, intergenerational settings, including the
contributions of Becker and Tomes \citep{becker1976child} on child
endowments and the altruistic dynastic model associated with Barro and
Becker \citep{barro1989fertility}. Next, I turn to macroeconomic and
unified growth models, notably Galor and Weil
\citep{galor2000population} and Galor and Moav \citep{galor2002natural},
which integrate the Q--Q mechanism into a general theory of demographic
and economic transformation.

Finally, I discuss more recent refinements that enrich the basic model
by incorporating credit constraints \citep{doepke2004accounting},
intra-household bargaining \citep{doepke2019bargaining}, and
multi-dimensional child quality
\citep{hoddinott2013economic, kalemli2002does}, with a special emphasis
on health and nutrition. The literature review focuses on how the Q--Q
framework has been applied to understand fertility and child investment
patterns, especially in developing country contexts where resource
constraints and health outcomes are paramount.

\subsection{Theoretical Background}

\subsubsection{Becker’s Static Model}

Becker's early work introduced an economic model of fertility and
treated children as durable goods that provide utility to parents but
impose costs \citep{doepke2015gary}. In Becker's 1960 model, a household
derives satisfaction from the number of children (\(n\)) and from the
quality of each child (\(q\)), alongside conventional consumption of
other goods (\(y\)). A simple representation is a utility function:

\[
U = U(n, q, y),
\]

with \(U\) increasing in each argument up to some satiation point. Here,
quality \(q\) can be thought of as the expenditure or investment per
child (e.g.~education spending, health care, nutrition), assumed for now
to be the same for each child. Parents face a budget constraint that
links quantity and quality: raising more children dilutes the resources
available per child. A prototypical budget constraint (in static form)
can be written as:

\[
p_y y + p_n n + p_q n q = I,
\]

where \(I\) is total family income (or full income), \(p_n\) represents
baseline, non-discretionary costs associated with each additional child
(e.g.~expenditures on food, shelter, or clothing that are incurred
irrespective of quality-enhancing investments), and \(p_q\) denotes the
marginal cost of investing in one unit of quality per child.The term
\(p_q n q\) captures total expenditure on quality for all children and
is linear in \(n\). As the number of children rises, parents must extend
any chosen level of \(q\) across a broader base, which amplifies the
total cost of quality. Conversely, the term \(p_n n\) implies that the
cost of an additional child rises with the quality level \(q\) already
chosen, since each child must meet a higher standard of care or
investment. For instance, a household that chooses to provide more
education or better health care per child incurs an additional burden
when it expands family size, as each child must receive the same
enhanced level of investment. Similarly, a larger family increases the
cumulative cost of quality, even if \(q\) remains fixed, due to the need
to replicate expenditures across more children. In short, the shadow
price of child quality increases with \(n\), and the shadow price of
child quantity increases with \(q\). The cost structure induces a mutual
dependence between quantity and quality, such that any adjustment along
one dimension alters the effective cost of the other.

Becker and Lewis (1973) formalize the mutually reinforcing nature of the
quantity--quality cost structure. An increase in \(n\) raises the total
cost required to sustain a given level of \(q\) for each child, while a
higher level of \(q\) raises the marginal cost associated with having an
additional child. For example, allocating more resources to education or
health per child increases the financial burden of expanding family
size. This interdependence links the two decisions directly. The
household cannot choose \(n\) and \(q\) in isolation; each choice alters
the marginal cost of the other.

Mathematically, the trade-off appears in the first-order conditions of
the household's optimization problem. Let \(\lambda\) represent the
Lagrange multiplier on the full-income constraint.

\[
\mathcal{L} = U(n, q, y) + \lambda \left( I - p_y y - p_n n - p_q n q \right).
\]

The first-order conditions are:

\[
\frac{\partial \mathcal{L}}{\partial n} = U_n - \lambda(p_n + p_q q) = 0, \quad
\frac{\partial \mathcal{L}}{\partial q} = U_q - \lambda p_q n = 0, \quad
\frac{\partial \mathcal{L}}{\partial y} = U_y - \lambda p_y = 0.
\]

Combining the first two yields:

\[
\frac{U_n}{U_q} = \frac{p_n + p_q q}{p_q n}.
\] This condition equates the marginal rate of substitution between
quantity and quality to the ratio of their full marginal costs. The
numerator rises with \(q\), and the denominator rises with \(n\). As one
choice increases, the relative cost of the other becomes higher. This
relationship induces substitution toward the less costly dimension. The
trade-off between quantity and quality arises from the structure of the
budget itself. It does not rely on specific assumptions about utility
curvature or intrinsic substitutability \citep{becker1973interaction}.

This formulation implies two core predictions. Firstly, although both
child quantity and child quality may rise with income, the household's
budget constraint can generate a negative relationship between income
and fertility. As income increases, total spending on children tends to
rise, but the allocation often favors quality over quantity. Becker
illustrated this with the analogy of durable goods: wealthier households
tend to upgrade the quality of a house or a car rather than acquire
additional units. In a similar way, higher-income families often direct
additional resources toward education, nutrition, or health per child.
Within the model, an income increase (\(\mathrm{d}I > 0\)) produces a
direct effect that makes children more affordable and an indirect effect
that discourages fertility. As \(q\) rises, the shadow price of an
additional child also rises. If the marginal utility from higher quality
exceeds that from larger family size, then the substitution effect
outweighs the income effect, leading to a lower optimal \(n\). This
mechanism offers a structural explanation for the demographic
transition: fertility tends to fall as households become richer, even
when preferences remain unchanged.

Furthermore, a similar logic applies to changes in the cost parameters
\(p_q\) and \(p_n\). A decline in \(p_q\), such as through a policy that
lowers the price of education or health care, increases \(q\) and raises
the marginal cost of quantity. This effect reduces optimal fertility. A
rise in \(p_n\), which may reflect higher child-rearing costs or a
greater opportunity cost of parental time, reduces the appeal of larger
families and can shift resources toward child quality. These outcomes
follow from the structure of the budget constraint, without requiring
any explicit preference for quality over quantity. Becker and Lewis
noted that these comparative static results align with observed
patterns. For example, increases in women's wages often reduce fertility
more than they reduce educational spending per child. This asymmetry
reflects the model's central feature: quantity and quality are linked
through their cost structure. An increase in one raises the marginal
cost of the other. The model explains how households make trade-offs
between the number of children and investments in each.

\subsubsection{Intergenerational Models: Altruism and Child Endowments}

While the early Q--Q models were static (one-period) representations,
subsequent contributions extended the framework to consider fertility
and child investment over multiple periods or even multiple generations.
The main development in this literature was the incorporation of
intergenerational human capital dynamics, where parents derive utility
not only from the number and quality of children in the present, but
also from the long-run outcomes of their offspring. These extensions
allowed child quality to evolve endogenously across time, rather than
being determined solely within a single period.

One of the earliest and most influential models of this kind was
proposed by \citet{becker1976child}. In their formulation, each child
enters the world with an exogenous endowment \(E\), which may reflect
factors such as cognitive ability (e.g., measured IQ or language
acquisition speed), early health status (e.g., birth weight or incidence
of neonatal complications), genetic predispositions (e.g., risk for
chronic illness, temperament, or neurodevelopmental traits), or family
background characteristics (e.g., parental education, household
stability, or neighborhood conditions). Furthermore, parents can augment
this endowment by investing resources \(q\) in the form of education,
nutrition, and other quality-enhancing inputs. The effective adult human
capital of the child might be expressed as \(H = E + f(q)\) (in a simple
additive form) or a multiplicative variant \(H = E \cdot f(q)\), where
\(f(q)\) is an increasing concave function of parental investment.

\citet{becker1976child} emphasized that variation in endowments \(E\)
can shape how parents allocate investments \(q\) across children. When
the productivity of investment increases with endowment, parents may
concentrate resources on children with higher \(E\), who are more likely
to convert additional investment into future success. In other cases,
parents may attempt to compensate for lower endowments by directing
greater investment toward disadvantaged children. Put simply,
child-specific variation in initial conditions affects not only outcomes
but also strategic parental choices. Because of this, the relationship
between income and the demand for child quality is not uniform. The
income elasticity of demand for quality may differ across households,
depending on the distribution of endowments within the family and across
the broader population.

Furthermore, \citet{becker1976child} showed that at low income levels,
much of what constitutes child quality comes from exogenous endowments,
i.e., factors like public education, neighborhood environment, or access
to basic healthcare that are not privately purchased. In these settings,
small increases in parental income may not lead to significant changes
in fertility or investment behavior. Since most of the child's future
outcomes are determined by the fixed endowment component, marginal
investment plays a smaller role. However, as income rises, private
resources become a larger part of what determines quality, and the
classic Q--Q trade-off begins to shape behavior. Parents begin to
allocate more income toward fewer children in order to enhance quality
through direct investment. Under certain theoretical conditions, such as
equal utility elasticities for quantity and quality, this framework
produces a non-monotonic relationship between income and fertility.
Fertility may decline as income rises at first, which reflects the
desire to invest more intensively per child. However, beyond some point,
once the marginal return to investment begins to flatten or saturate
relative to the fixed endowment, fertility may increase again.

The U-shaped prediction emerges only under specific assumptions, and its
validity depends on both the shape of the utility function and how
endowments relate to parental background. More broadly,
\citet{becker1976child} enriched the Q--Q framework by incorporating
elements that reflect real-world variation. They argued that not all
differences in child outcomes are the result of deliberate parental
choice. Random factors, biological traits, and socioeconomic settings
play a role. In this light, public policies, such as subsidized
schooling, early childhood programs, or universal healthcare, can
influence private fertility and investment decisions by shifting the
effective value of \(E\) across the population. If government programs
raise the floor for child endowments, then even low-income parents can
achieve better outcomes without large private sacrifices. These
policy-induced shifts in \(E\) alter the perceived return to having more
children or investing more per child.

\citet{becker1976child} also considered the possibility that endowment
is not randomly assigned but may vary systematically with income.
Higher-income households may produce children with higher \(E\) due to
better maternal nutrition, access to prenatal care, lower exposure to
environmental risk, or assortative matching on traits associated with
educational or occupational success. In these families, not only are the
resources available for investment greater, but the potential gains from
investment may also be higher, because children are better positioned to
benefit from those inputs. This interaction deepens the divide between
high- and low-income households, making it harder for disadvantaged
families to catch up. As a result, inequality can persist or even widen
across generations.

Finally, the model provides a mechanism for understanding how
imperfections in credit markets can lead to persistent disadvantages. If
parents with low income and low-\(E\) children cannot borrow to finance
quality-enhancing investment, then the next generation begins life with
the same disadvantage. Without external intervention or structural
change, this loop continues, which results in a pattern where poor
families remain poor and rich families accumulate further advantage. The
Becker-Tomes framework thus connects household-level decisions to bigger
questions about the intergenerational transmission of human capital.

Parallel to Becker and Tomes's static analysis of endowments, another
strand of the literature developed a fully dynamic version of the Q-Q
model by incorporating parental altruism toward children's welfare. In
this framework, introduced by \citet{barro1989fertility}, parents care
not only about the number and quality of their children but also about
the utility their descendants will enjoy in the future. Altruism in this
context means that parents treat their children's utility as part of
their own, thus extending the household's objective across generations.
For example, a parent may reduce personal consumption to pay for a
child's schooling, motivated not just by the child's immediate benefit
but by the satisfaction the parent gains from the child's long-term
success. This leads to a formulation of dynastic utility, where the
household's objective spans infinitely many periods and takes the form
of a recursive altruistic structure. A representative formulation is

\[
U_{0} = \sum_{t=0}^{\infty} \beta^{t}\,u(c_{t}, n_{t}),
\]

where \(c_t\) denotes the consumption of the \(t\)-th generation,
\(n_t\) the number of children, and \(\beta \in (0,1)\) the
intertemporal discount factor. Given this structure, having an extra
child \(n_t\) enters utility positively, but each child is assumed to
receive the same utility as the parent if raised at a comparable
standard of living. As a result, parents confront an intertemporal
trade-off: having more children expands the number of future utility
streams but also stretches current resources, since each child requires
support. This trade-off gives rise to an Euler equation for optimal
fertility choice, analogous to an optimal growth condition.

An implication of the dynastic model is that fertility decisions are
sensitive to macroeconomic conditions, such as the interest rate or the
rate of return on capital. A rise in interest rates increases the
opportunity cost of channeling resources into children rather than
saving, which tends to reduce current fertility---a substitution effect
across generations. At the same time, higher returns make future
generations wealthier, and this anticipated prosperity enters the
utility calculations of parents in more complex ways.
\citet{barro1989fertility} demonstrated that the model can account for
observed fertility responses to economic fluctuations and policy
interventions It can also explain historical phenomena such as postwar
baby booms and subsequent fertility declines through shifts in returns
or labor‐market opportunities.

In many dynastic models, child quality appears indirectly, often through
the child's future human capital or income. One variant assumes parents
value the aggregate human‐capital stock of their children. This
specification, combined with altruism, produces a similar trade‐off:
concentrating resources in fewer children raises each child's human
capital, which raises the dynasty's long‐run welfare. These
intergenerational extensions link micro‐level fertility decisions to
macroeconomic outcomes. By the late 1980s, work by Becker, Barro, and
others had recast fertility as an endogenous choice that interacts with
capital accumulation, income distribution, and policy. This laid the
foundation for unified growth theories, which view the quantity--quality
mechanism as central to demographic transition and long-run development.

\subsubsection{Unified Growth Models}

The unified growth theory, developed in the late 1990s and 2000s
(notably by Oded Galor and co-authors), seeks to explain in one
framework the entire sweep of economic development -- from Malthusian
stagnation, through the demographic transition, to modern growth. A
central puzzle it addresses is why fertility rates, which were
historically high and invariant to income in the Malthusian era, began
to decline sharply in tandem with industrialization and rising incomes,
eventually stabilizing at much lower levels in developed economies. The
Q--Q trade-off provides a key part of the answer in these models.
\citet{galor2000population} and \citet{galor2002natural} explicitly
incorporate parental choices about the quantity and quality of children
and show how changes in the economic environment alter those choices and
trigger demographic transitions.

In \citet{galor2000population}'s model, for instance, technological
progress gradually increases the return to human capital, especially in
skilled occupations. In the early stages, when production relies on
basic tools and techniques, unskilled labor holds more value. Under
these conditions, parents have little reason to invest in formal
schooling. Children are expected to contribute economically through
agricultural work, domestic tasks, or low-skill jobs in workshops and
factories. Fertility remains high because children impose a low
financial burden and generate immediate returns. As technology becomes
more advanced---such as during the Industrial Revolution---the earnings
gap between skilled and unskilled labor widens. Education begins to
offer significant advantages in the labor market. In response, parents
adjust by having fewer children and placing greater emphasis on each
child's development, including school attendance and better health care.

Evidently, industrialization raises the economic value of skilled labor,
which alters household incentives. As returns to education increase,
parents begin to favor investments in child quality over child quantity.
This shift results in declining fertility because families choose to
have fewer children and allocate more resources to each. The feedback
effect is significant: higher educational investment raises productivity
in the next generation, which in turn accelerates technological
advancement and further increases the returns to human capital. Over
time, the economy moves from a state of high fertility and low growth to
one characterized by low fertility and sustained growth.

Furthermore, \citet{galor2002natural} introduced an evolutionary
refinement to the unified growth framework by accounting for
heterogeneity in parental preferences. During the Malthusian period,
some families placed greater emphasis on child quality, such as
education, while others prioritized quantity. In a stagnant economy with
limited returns to education, high-fertility lineages maintained a
numerical advantage and suppressed average human capital. As
technological progress increased the returns to education, families that
valued quality gained an economic edge. Their children acquired more
human capital and achieved higher income and survival rates. These
advantages allowed such families to grow in relative size. Over time,
this process resembled a form of evolutionary selection, gradually
favoring quality-oriented parental types and shifting the population
toward greater emphasis on child human capital. These dynamics
strengthened the shift from high-fertility, low-education regimes to
low-fertility, high-investment family structures. In formal
overlapping-generations models, \citet{galor2002natural} demonstrate
that this evolutionary adaptation accelerates the demographic
transition. Their framework accounts for the rapid and widespread drop
in fertility once it takes effect. Higher returns to human capital push
parents to favor quality, while preferences for quality begin to
dominate within the population. These forces support the emergence of a
low-fertility, high-investment equilibrium and establish a unified
explanation for both economic development and demographic change.

Importantly, unified growth models identify several complementary
mechanisms that reinforce the basic Q--Q trade-off during development.
One is the decline of child labor. As the economy modernizes, the value
of child labor falls, both because legal reforms often restrict child
labor and because parents realize the earnings their children could make
as unskilled laborers are paltry compared to the potential returns if
those children instead spend time in school. \citet{hazan2002child}
formally show that when child labor becomes less profitable relative to
adult (skilled) labor, parents further reduce fertility and invest more
in each child's education. Historical evidence from England, for
instance, indicates that during industrialization the wages of children
(relative to adults) dropped significantly, especially in skilled
families, and this was accompanied by parents pulling children out of
work to send them to school. \citet{galor2006human} even argue that
capitalist industrialists supported public education laws and child
labor bans as a way to increase the human capital of the workforce,
inadvertently hastening the fertility transition.

Another mechanism is the rise in life expectancy and child survival.
Improvements in sanitation, nutrition, and medical knowledge in
developing societies led to more children surviving to adulthood. While
the earliest unified growth models treated mortality as exogenous or
ignored it, later research demonstrated that declining child mortality
can trigger lower fertility as well-- parents no longer need ``extra''
births for insurance once they are confident their existing children
will survive. In other words, increased child survival and the
quality--quantity trade-off are complementary explanations for fertility
decline that operate in tandem. When fewer births are lost to disease,
parents can achieve a desired number of surviving offspring with fewer
total births, and they tend to reallocate effort into each child's
health and education.

The overall effect is a reinforcing cycle: better health raises the
returns to schooling (healthier children can learn more effectively and
have longer working lives), which further encourages educational
investments and reduces fertility. Indeed, Galor notes that human
capital should be interpreted broadly to include health as well as
schooling; in unified growth theory, improvements in nutrition and
physical well-being were crucial to making labor more productive and
thus were part and parcel of the rise in demand for human capital.

The unified growth literature places the Q--Q model within a more
general account of economic and demographic change. In this framework,
higher income or stronger returns to child quality reduce fertility and
help shift economies from stagnation toward sustained growth. Several
mechanisms support this transition, such as a fall in child labor, a
drop in child mortality, and a shift in parental priorities. These
models explain not only the presence of a quantity--quality trade-off
but also its rising influence at a specific point in history. The
evidence supports these claims: countries that saw earlier increases in
returns to education experienced earlier fertility decline, while delays
in reforms, such as public education or health access, corresponded to
prolonged high fertility. As a result, the Q--Q mechanism forms a key
component of unified growth theory.

\subsection{Recent Refinements to the Q-Q Model}

Contemporary research has further refined the quantity--quality model by
relaxing some of its initial simplifying assumptions. Three important
extensions involve (1) capital market imperfections that constrain
parents' ability to invest in child quality, (2) intra-household
conflict and bargaining between mothers and fathers over fertility
choices, and (3) recognition that child quality is multi-dimensional,
which extends beyond schooling to include health, nutrition, and other
facets of human capital.

To begin, I examine how credit constraints can give rise to poverty
traps. The canonical quantity--quality (Q--Q) framework assumes that
parents can reallocate resources freely; borrowing against future
earnings to finance schooling or health investments whenever the
expected return is high. In practice, especially in low-income settings,
credit markets function imperfectly: poor households typically cannot
secure loans to cover children's education or medical costs even when
such investments would yield substantial future gains. This market
failure magnifies the Q--Q trade-off. Becker, Lewis, and Willis--already
noted by \citet{grawe2008quality} as emphasizing ``resource
limitations''--implicitly recognized this issue, but contemporary models
make it explicit by imposing a borrowing constraint. Parents must fund
childrearing from current income alone; they cannot collateralize a
child's future wages to pay today's school fees. Consequently, when
income is low, each additional birth directly reduces the attainable
quality per child, potentially trapping families in a
low-income--high-fertility equilibrium.

Formal models show that ``the quality--quantity trade-off arises from a
binding credit constraint that prevents parents from borrowing against
future child income.'' Empirical work supports this mechanism.
\citet{kremer2002income} and \citet{de2003inequality} document that
countries facing tighter liquidity constraints tend to display higher
fertility and lower educational attainment, consistent with
liquidity-constrained parents favoring quantity over quality.
Cross-country evidence likewise indicates that where financial frictions
are more severe, the negative correlation between fertility and
schooling is stronger. Theoretically, introducing a borrowing limit can
generate multiple steady states: one with low fertility and high
investment when incomes suffice to cover quality costs, and another with
high fertility and minimal investment when they do not. Policy
instruments such as education subsidies or conditional cash transfers
effectively relax these constraints, which nudge households toward the
low-fertility, high-investment equilibrium. In short, incorporating
credit market imperfections deepens the explanatory power of the Q-Q
model: economic growth alone may not reduce fertility if households
remain too cash-poor to afford schooling, whereas targeted
quality-enhancing transfers can catalyze both demographic and
human-capital transitions.

A further refinement of the quantity--quality (Q--Q) model considers the
question of \emph{who} within the household makes fertility and child
investment decisions. The original Beckerian framework adopts a unitary
model of the family, and assumes a single utility function and complete
agreement between spouses over optimal fertility \(n\) and child quality
\(q\). In practice, however, empirical evidence reveals significant
heterogeneity in preferences between household members along gender
lines. According to \citet{oppenheim1987impact, thomas1990intra}, for
instance, men often desire more children than women and may differ in
their willingness to invest in each child's education or health. These
discrepancies have motivated game-theoretic models of intra-household
bargaining, in which fertility and investment outcomes reflect the
relative influence of each parent's preferences.

In such models, the mother is typically assumed to have a stronger
preference for child quality---such as health and schooling---while the
father may favor either more children or alternative uses of household
resources. The resolution of these conflicting preferences depends on
bargaining power, which can be shaped by income contributions, legal
rights, cultural norms, or access to external resources. When the
mother's bargaining power increases, theoretical models predict a shift
toward lower fertility and higher per-child investment, holding other
factors constant. This prediction is consistent with empirical findings:
\citet{iyigun2007endogenous, doepke2019bargaining} show that greater
female empowerment---via education or labor force
participation---correlates with reduced fertility and increased
investment in child human capital.

Mathematically, the household's first-order condition for fertility in a
bargaining model can be written as:

\[
\alpha \frac{\partial U_{\text{wife}}}{\partial n} + (1 - \alpha) \frac{\partial U_{\text{husband}}}{\partial n} = \lambda(p_n + p_q q),
\]

with a corresponding condition for \(q\). Here, \(\alpha \in [0,1]\)
represents the wife's bargaining weight. When \(\alpha\) increases, the
composite marginal utility of additional children typically
decreases---especially if the wife prefers fewer children---leading to
lower equilibrium fertility and a shift along the Q--Q frontier toward
higher quality.

Recent work also explores dynamic bargaining, in which spouses negotiate
sequential decisions over time, potentially leading to strategic
behavior (e.g., one partner may accelerate or delay subsequent births).
Although these models introduce complexity, their core implication for
the Q--Q framework is clear: household power dynamics fundamentally
shape the trade-off between child quantity and quality. In societies
where women have limited decision-making autonomy---due to lack of
access to contraception, or social norms---fertility tends to remain
high and per-child investment low, which stalls demographic transition.
Conversely, when women gain bargaining power---through legal reforms,
labor market participation, or targeted transfers---the household often
reallocates resources toward fewer but higher-quality children.

This theoretical insight is corroborated by policy experiments. For
instance, cash transfer programs directed to mothers consistently lead
to greater spending on children's health, education, and nutrition
compared to equivalent transfers given to fathers
\citep{duflo2003grandmothers, thomas1990intra} . Such outcomes support
the hypothesis that mothers place higher weight on child quality, and
that who controls the purse strings matters deeply. In sum, introducing
intra-household bargaining into the Q--Q model enriches its explanatory
scope: it highlights how family structure and power asymmetries---not
just income levels or prices---generate variation in fertility and human
capital outcomes across and within societies.

\subsection{Health and Nutrition in the Q–Q Model}

The original Q--Q models often used a single catch-all variable for
child quality, typically thought of as education or ``expenditure per
child.'' Recent work emphasizes that child quality is multi-faceted, and
that parents make trade-offs along several dimensions of
investment---cognitive development, health, nutrition, etc. This is
particularly salient in developing countries, where basic health and
nutrition are pressing concerns alongside schooling.

The theoretical question is how these dimensions interact with the
quantity decision. If parents allocate a budget across, say, schooling
\(q_{\text{edu}}\) and nutrition/health \(q_{\text{health}}\) for each
child, then having more children forces cutbacks in both dimensions
(unless parents reallocate across them). In some models, health and
education are complementary: a healthier child benefits more from
education, and an educated mother might raise a healthier child. This
complementarity can amplify the Q--Q trade-off---investing in one
dimension (health) increases the returns to investing in the other
(education), so a high-quality strategy becomes more focused on fewer
children.

On the other hand, if one dimension has diminishing returns more quickly
than another, parents might prioritize achieving a threshold level of
health for all children before adding more education, which creates a
nonlinear effect on fertility. One especially important aspect of health
in the Q--Q framework is child survival. The probability that a child
survives to adulthood effectively multiplies the utility of having that
child. Historically, high child mortality led to a strategy of
``quantity for insurance,'' where parents had additional births to
ensure survivorship. As mortality falls due to public health
improvements, parents can shift toward quality without risking
childlessness \citep{kalemli2002does, kalemli2000mortality}.

\citet{kalemli2002does} developed a stochastic model in which fertility
choices are made under uncertainty about child survival. Her results
show that declining mortality causally reduces fertility and increases
educational investment per child, in line with the Q--Q trade-off. In
unified growth models, declining child mortality reinforces the
demand-for-human-capital channel of the demographic transition
\citep{galor2004physical}.

Beyond survival, nutritional status is a critical indicator of quality
in many low-income settings. Parents often face choices about how to
allocate food or medical care, especially under resource scarcity.
Empirical evidence supports the Q--Q model in this context: children
with many siblings often exhibit poorer nutritional outcomes. A recent
study using Vietnam Young Lives data finds that an additional sibling
reduces height-for-age and weight-for-age z-scores by about 0.3 standard
deviations on average \citep{chen2021population} This effect is
substantial, suggesting that children in larger families tend to be
shorter and lighter than their peers from smaller families, likely due
to the more limited distribution of healthcare and parental attention

Similar patterns emerge in sub-Saharan Africa
\citep{bishwakarma2019first} and South Asia , where first-born children
experience slower growth once younger siblings arrive---again pointing
to intrahousehold trade-offs. Theoretically, one can extend the Becker
model with a nutrition production function, where a child's health \(H\)
depends on food \(F\) and medical care \(M\), such that:

\[
H = g(F, M)
\]

and \(H\) enters either utility directly or the child's future
productivity. Parents then choose \(n\), \(F\), \(M\), and possibly
schooling per child. Larger \(n\) reduces \(F\) and \(M\) per child,
lowering \(H\); if parents place high value on \(H\), or if \(H\)
enhances returns to education, they will opt for smaller families.

Importantly, improvements in nutrition (through income growth or public
programs) can first lead to both better health and higher fertility
(since healthier women can bear more children). But over time, as
standards of living rise, better nutrition increases the returns to
investing in fewer, healthier children. Economic historian
\citet{fogel1994economic} argues that Europe's demographic transition
was partly driven by better diets: improved nutrition raised the
productivity of educated workers, which made education more valuable and
shifted family preferences toward quality.

These propositions are important for modern development policy. Family
planning programs that encourage lower fertility without improving child
health and education may have limited long-term effects. Conversely,
health and nutrition interventions---such as vaccinations or food
supplementation---can enhance the returns to education and induce
parents to reduce fertility voluntarily. The extended Q--Q framework
thus serves as a unifying tool to understand how diverse
interventions---ranging from school fees to nutrition programs---shape
long-run development trajectories.

\subsection{Philippine Evidence on the Q–Q Trade-off}

Early empirical work in the Philippines provided suggestive support for
this theory. For example, an influential study by
\citet{horton1986child} used Philippine household data to jointly
examine fertility and child nutrition and treated nutritional status as
a measure of child quality. Horton found that better-educated mothers
and fathers tended to choose smaller families and achieved
better-nourished children--evidence of substitution away from
``quantity'' toward child ``quality.'' Notably, she observed strong
birth-order effects on nutrition (first-born children faring better than
later-born), which hint that parents may not evenly distribute quality
investments among all children. This early work indicated a
quantity--quality (Q--Q) trade-off in Philippine families, though it
largely documented correlations rather than definitive causation.

Subsequent studies in the Philippines have tackled the critical issue of
causality, using innovative research designs to isolate exogenous
changes in family size. One seminal contribution is
\citet{orbeta2010number}, who examined the impact of family size on
children's schooling using a nationally representative survey.
Recognizing that family size is endogenous (parents' fertility choices
may reflect unobserved preferences or constraints), Orbeta employed an
instrumental-variable (IV) approach grounded in Becker's framework.
Specifically, he used the sex composition of the first two children as
an instrument for having additional children -- an approach pioneered by
\citet{angrist1996children} -- leveraging the fact that Filipino parents
often desire a mixed-gender sibset (and are more likely to have a third
child if the first two are the same sex) \citep{vicerra2013fertility}.
This strategy aims to generate random-like variation in family size
uncorrelated with parental characteristics.

The IV estimates confirmed a significant negative causal effect of
higher fertility on educational outcomes. \citet{orbeta2010number} found
that each additional child in the household reduced the proportion of
school-age children (6--24 years) attending school by roughly 19\% of
the baseline attendance rate. The trade-off was especially pronounced at
higher education levels: for example, the estimated drop in school
attendance was about 26\% at the secondary level and 57\% at the
tertiary level for each additional sibling. These are sizable effects,
implying that children from large families are substantially less likely
to remain in school, presumably due to tighter household budget
constraints or diluted parental attention. Moreover, the burden of the
trade-off appeared regressive: Orbeta's results showed much larger
schooling deficits from an extra child in poorer households than in
richer ones. For instance, in the poorest quintile, an additional
sibling reduced school attendance by an estimated 24\% (for ages 6--24),
compared to a 16\% reduction in the richest quintile. This regressive
pattern aligns with Becker's theory that resource constraints bind more
tightly for low-income families, which makes the Q--Q trade-off more
acute. In summary, Orbeta's study -- the first in the Philippines to
account for fertility endogeneity -- provides robust evidence that
increases in family size cause significant declines in child educational
attainment and validates the quantity--quality trade-off in this
context. Large family size thus emerges as one mechanism contributing to
poverty, by impeding children's human capital accumulation in the
Philippines.

Further compelling evidence comes from a natural experiment studied by
\citet{dumas2019sex}. They exploit a unique policy shock in metropolitan
Manila to identify the trade-off. In 1998, the Mayor of Manila city
imposed a sudden ban on modern contraceptives in public facilities,
drastically curtailing access to family planning for residents of Manila
city (but not in surrounding municipalities). Dumas and Lefranc use this
policy as a quasi-experiment: comparing families in Manila city (treated
by the ban) to similar families in other cities unaffected by the ban
before and after 1998. This difference-in-differences design, coupled
with the fact that older mothers were naturally less fecund during the
ban, isolates an exogenous fertility increase.

The results are striking. The contraceptives ban led to a significant
rise in births and family size in Manila city relative to the control
areas. Correspondingly, children born in Manila during the ban era
experienced a sizable decline in educational attainment -- clear
evidence of a Q--Q trade-off precipitated by the shock. In the authors'
words, the policy-driven increase in family size ``provide{[}s{]}
evidence of a quality--quantity trade-off'': larger families, forced by
the ban, resulted in lower schooling outcomes per child. By leveraging
an actual anti-contraception policy, this study offers causal
confirmation that constraints on family planning can worsen child
quality. Together with Orbeta's findings, it reinforces the relevance of
Becker's hypothesis in the Philippine setting: whether via unintended
fertility (Manila's case) or ordinary variation in sibling sex mix, more
children mean fewer resources per child and thus worse education
outcomes. Consistent with theory, Filipino parents facing exogenous
increases in quantity were unable to maintain the same level of quality
per child.

Beyond education, researchers have also examined health and other child
outcomes in relation to family size. Evidence generally suggests the
Q--Q trade-off extends to child health and nutrition. For instance,
\citet{horton1986child} already hinted that large families may
compromise child nutrition for later-born siblings. More recent regional
research resonates with this. \citet{hatton2018fertility} analyze the
effect of fertility on child height (a long-run health indicator) using
longitudinal data from Indonesia -- a neighboring Southeast Asian
country with comparable developmental challenges. They address
endogeneity by exploiting Indonesia's family planning program rollout
and exposure to mass media as instruments for fertility. The authors
find a significant negative impact of family size on child health: each
additional sibling is associated with about a one-third standard
deviation reduction in a child's height-for-age Z-score, after
controlling for other factors. This health penalty from having more
children is strongest in low-education households and appears in both
urban and rural areas. Such findings mirror the Philippine evidence that
the harms of large family size are most pronounced among disadvantaged
families. In economic terms, poorer parents with many children struggle
to provide adequate nutrition and schooling to all, which highlight the
equity dimension of the Q--Q trade-off.

\subsection{Family Size and Child Outcomes in Southeast Asia}

The Philippine experience is echoed in other ASEAN countries, where
researchers have probed the quantity--quality trade-off with diverse
outcomes and methods. In Vietnam, for example, rapid fertility decline
alongside rising education led to questions about a Q--Q mechanism.
\citet{anh1998family} documented a negative correlation between family
size and children's school enrollment in Vietnam, though their analysis
could not fully establish causality.

More rigorously, \citet{dang2016decision} used distance to the nearest
family planning center as an instrument to study Vietnamese households'
investments in education. They introduced a novel measure of child
quality---spending on private tutoring, a prevalent form of educational
investment in Vietnam---alongside traditional indicators like schooling
expenses. The IV estimates confirmed that children with more siblings
receive significantly lower educational investments from their families.
In particular, Vietnamese families of larger size spent less on each
school-age child's schooling and tutoring, even after controlling for
community factors. This effect was robust across different definitions
of family size and model specifications, indicating that Vietnamese
parents do indeed trade off quantity for quality when faced with
resource constraints. Such evidence aligns squarely with Becker's model:
as Vietnamese family size increases, per-child education spending falls;
parents are prioritizing ``quality'' less when they have more offspring.

Indonesia and Thailand show similar patterns. Demographic research in
Thailand during its fertility transition found that large families had
markedly worse educational outcomes. \citet{knodel1990family}, studying
Thai data in the 1990s, observed that once family size exceeded about
4--5 children, the likelihood of a child progressing to or staying in
secondary school dropped precipitously compared to smaller families.
Although these early Thai studies were based on correlations, they
strongly suggested that limited family resources were being spread thin
in big families and hurt children's schooling attainment.

In Indonesia, cohort analyses and natural experiments reinforce the
trade-off. \citet{maralani2008changing} showed that the relationship
between sibship size and schooling evolved from neutral or even positive
for older cohorts (born when education opportunities were limited) to
negative for more recent cohorts, consistent with a growing importance
of education in a modernizing economy. More concretely, the
aforementioned study by \citet{hatton2018fertility} in Indonesia
provides causal evidence that mirrors the Philippine findings in health
and nutrition. Likewise, an analysis of Indonesian census data
\citep{feng2021effect} found that having additional siblings
significantly lowers children's educational attainment once birth order
effects are accounted for, paralleling results from China and Vietnam.

These regional studies share a commonality: in resource-constrained
settings of Southeast Asia, increased child quantity tends to come at
the expense of child quality, be it years of schooling, academic
spending, or health status. The consistency of this pattern---across
countries with different cultures and policies---highlights the
fundamental economic logic identified by \citet{becker1973interaction}.
Parents with finite resources face difficult choices, and many appear to
balance quantity and quality in a way that confirms the trade-off
hypothesis.

Despite these similarities, there are some noteworthy nuances and gaps
in the ASEAN literature. One is the role of public policy and
development level. Evidence suggests that the Q--Q trade-off may be
mitigated in contexts with strong public support for education and
health. For instance, studies in developed countries (e.g.~Israel,
Norway) often find little or no trade-off once factors like birth order
are accounted for
\citep{black2005more, kristensen2010educational, angrist2010multiple}.
In Southeast Asia, however, public education quality and social safety
nets are still developing, and the cost of raising children (education
fees, food, etc.) is largely borne by families themselves
\citep{oecd2024sigi}. This may explain why the trade-off emerges so
clearly in the Philippines, Vietnam, Indonesia, and Thailand.

Another nuance is methodological: more recent studies employ credible
identification strategies (IVs, twins, policy shocks) and consistently
find a causal negative effect of family size on child outcomes, whereas
older studies without such controls sometimes found weaker effects or
none at all. This highlights the importance of accounting for
endogeneity.

Finally, there remain gaps for future research. Most ASEAN studies focus
on education and early-life health indicators as measures of child
quality; there is relatively little evidence on long-term outcomes such
as children's eventual earnings or income in adulthood. It is not yet
fully clear whether the schooling and health disadvantages observed in
larger families translate into significantly lower adult productivity or
income---a link that \citet{becker1976child} theorized but which could
be explored further in this region.

Additionally, while the trade-off appears pervasive, its magnitude can
vary: for example, the penalty of an extra child may be larger in poorer
rural areas than in urban or wealthier settings, suggesting that local
context (poverty, gender norms, access to services) can modulate the
trade-off. Comparative studies across ASEAN are still somewhat limited,
and a common challenge is disentangling related factors like birth
order, sibling composition, and parental preferences.

Nonetheless, the prevailing evidence from the Philippines and its
regional neighbors strongly supports Becker's quantity--quality
conjecture. As families have fewer children, they appear to invest more
in each child's education---investments crucial for human capital
development and economic growth. Conversely, high-fertility households
risk under-investing per child, which reinforces cycles of poverty and
inequality. This literature thus provides an important empirical
foundation for policies in the Philippines and ASEAN---from family
planning programs to education subsidies---that aim to ease the
quantity--quality trade-off and help families achieve both manageable
size and better outcomes for the next generation.

\section{Conceptual Framework and Hypotheses}

This study builds upon the foundational framework of the
quantity--quality (Q--Q) trade-off, as articulated in
\citet{becker1960economic} seminal work and later refined by
\citet{becker1973interaction}. Within this theoretical model, fertility
and child quality are jointly determined under a full-income constraint,
such that any increase in the number of children---absent a commensurate
expansion in resources---necessitates a reduction in the average
investment per child. While prior empirical applications of this
framework have largely centered on educational attainment, child quality
may also be understood along biological dimensions, especially along
nutritional status, which emerges earlier in the life course and may be
less subject to later remediation.

The current analysis focuses exclusively on early-life nutrition as a
proxy for child quality. Nutrition, unlike cognitive or educational
outcomes, is characterized by low substitutability and a narrow temporal
window during which investments are most productive. The trade-off is
thus expected to manifest strongly in settings where fertility rises
exogenously while household resources remain constrained. In such
environments, families facing an unexpected increase in sibship size may
be forced to reallocate scarce food, time, and health inputs across a
larger number of dependents, potentially resulting in lower nutritional
attainment for each child.

To formalize this relationship, consider a household with total income
\(I\), choosing nutritional investments \(\kappa_j\) for each of \(n\)
children, along with consumption \(Z\) unrelated to child quality:

\[
I = \sum_{j=1}^{n} \kappa_j + Z.
\]

In the presence of a fertility shock, holding \(I\) constant, the
average \(\kappa_j\) must fall unless offset by changes in parental
behavior or external transfers. If nutritional investments per child
decline, and if such inputs are critical for growth, observable deficits
in height-for-age (HAZ) or weight-for-height (WHZ) may emerge. The first
hypothesis emerging from this framework is that an exogenous increase in

fertility leads to a reduction in per-child nutritional investment,
particularly in low-income, urban households where slack in the family
budget is limited. A second, related hypothesis posits that this
reduction in inputs translates into measurable nutritional deficits
among young children, specifically in the form of lower anthropometric
outcomes such as height-for-age (HAZ) and weight-for-height (WHZ)
z-scores. These two hypotheses---concerning, respectively, the
behavioral reallocation of resources and its biological
consequence---jointly seek to identify a causal link between fertility
shocks and early-life deprivation.

\section{Data and Empirical Strategy}

This study constructs a harmonized, child-level analytic dataset using
five rounds of the Philippine Demographic and Health Survey (DHS): 1993,
1998, 2003, 2008, and 2013. The DHS is a nationally representative,
cross-sectional survey series conducted by the Philippine Statistics
Authority in collaboration with ICF International. Each round follows a
stratified two-stage cluster design and collects standardized
information on child health, fertility, maternal characteristics, and
household infrastructure. For each wave, I merge the child-level (KR),
mother-level (IR), and household-level (HR) files using cluster,
household, and maternal identifiers, ensuring that each observation
corresponds to a live birth occurring within five years of the survey.
The analytic sample is restricted to children aged 0--59 months to
standardize developmental exposure windows and conform to the World
Health Organization (WHO) reference growth standards.

The primary outcomes are height-for-age (HAZ) and weight-for-height
(WHZ) z-scores, internationally recognized proxies for chronic and acute
malnutrition, respectively. These anthropometric indicators are cleaned
following DHS protocols: biologically implausible values are excluded,
and remaining z-scores are rescaled from hundredths to continuous units.
In addition to anthropometry, the child record includes a suite of
health-related variables. These include indicators for whether the child
has received basic vaccinations---such as BCG, DPT, polio, and
measles---as well as whether the child experienced episodes of diarrhea,
fever, or respiratory illness in the two weeks prior to the survey.
These supplementary health measures serve as intermediate outcomes in
secondary analyses to capture short-run morbidity and preventive health
investment.

Maternal-level information is drawn from the individual recode file.
Core covariates include maternal age, completed years of education, and
total number of children ever born (parity), each of which captures
socioeconomic status and fertility history. I additionally include
maternal body mass index (BMI) and anemia status, which serve as both
controls and potential mediators of child health. These indicators
reflect the mother's nutritional status and physiological capacity to
support healthy fetal and postnatal development. In robustness checks, I
examine whether declines in maternal nutritional status following the
fertility shock account for observed changes in child anthropometry.

Household-level characteristics are used to control for environmental
and infrastructural determinants of health. Specifically, I include
household size, urban residence, wealth quintile, access to electricity,
type of toilet facility, whether the toilet is shared, and source of
drinking water. These variables proxy for household-level resources and
exposure to disease environments. In extension analyses, I interact
treatment status with these environmental indicators to assess whether
the nutritional consequences of the fertility shock are magnified under
infrastructural deprivation.

The treatment is defined by exposure to Executive Order No.~003, which
was issued by the Manila City government in January 2000 and effectively
suspended the public distribution of modern contraceptives. This policy
change resulted in a localized increase in fertility that was not
mirrored in surrounding municipalities within the National Capital
Region (NCR). Children born in Manila City after the implementation of
the policy are classified as treated, while children in other NCR cities
serve as the control group. This spatially and temporally bounded
intervention provides quasi-experimental variation that facilitates
identification of causal effects.

The empirical approach adopts a difference-in-differences framework that
compares trends in child nutritional outcomes between treated and
control cities before and after the policy change. The identifying
assumption is that, absent the policy, nutritional trajectories in
Manila and non-Manila NCR cities would have evolved in parallel. This
assumption is supported by visual inspection and pre-trend placebo tests
using the 1993 and 1998 waves. The 1998 DHS serves as the primary
pre-policy baseline, while the 2003 survey captures short-run effects.
The 2008 and 2013 waves are used in robustness checks to assess medium-
and long-run persistence of nutritional deficits. All specifications
include city and birth-year fixed effects, and standard errors are
clustered at the city-year level to account for within-cluster
correlation.

The empirical strategy exploits the 2000 issuance of Executive Order
No.~003, which effectively banned the distribution of modern
contraceptives in public health centers across Manila City. This abrupt
and localized withdrawal of contraceptive access resulted in a fertility
increase concentrated within the city, but did not affect surrounding
municipalities within the National Capital Region (NCR). Children born
in Manila following the policy are considered exposed to a fertility
shock, while children in other NCR cities over the same period serve as
the control group. The analytical sample is restricted to children under
five years old at the time of each survey round to ensure accurate
measurement of early-life nutritional outcomes.

I estimate the following difference-in-differences specification:

\[ \text{Nutrition}_{ict} = \alpha + \beta \cdot (\text{Post}_t \times \text{Manila}_c) + X_{ict}'\gamma + \lambda_c + \tau_t + \varepsilon_{ict} \]

In this equation, \(\text{Nutrition}_{ict}\) refers to either
height-for-age (HAZ) or weight-for-height (WHZ) for child \(i\) in city
\(c\) born in year \(t\). The indicator \(\text{Post}_t\) equals 1 if
the child was born after the implementation of Executive Order No.~003
in 2000, and \(\text{Manila}c\) equals 1 if the child resides in Manila
City. The interaction term captures the effect of being born in Manila
after the contraceptive policy was enacted. The vector \(X{ict}\)
includes controls for child age in months, sex, birth order, maternal
age, education, parity, body mass index (BMI), anemia status, household
wealth quintile, household size, electricity access, water source,
toilet type, shared sanitation status, and urban residence. City fixed
effects \(\lambda_c\) control for time-invariant characteristics at the
locality level, while year fixed effects \(\tau_t\) absorb aggregate
time shocks. The coefficient of interest, \(\beta\), represents the
average treatment effect of the policy-induced fertility shock on child
nutrition. Standard errors are clustered at the city-year level to
account for potential autocorrelation within geographic units over time.

To support the internal validity of the design, I use Philippine Census
microdata from 1990, 1995, 2000, 2007, and 2010 to confirm the magnitude
and timing of the fertility shock. The 1995 and 2007 waves provide clean
pre- and post-policy benchmarks, while the 2000 census---although
temporally proximate to the policy---is retained for robustness checks
that test the sensitivity of estimates to the exact onset of fertility
changes. A placebo comparison between 1990 and 1995 shows no divergence
in fertility trends between Manila and control cities prior to the
policy, strengthening the credibility of the parallel trends assumption.

To supplement the DHS and Census data, I incorporate the Annual Poverty
Indicators Survey (APIS), a nationally representative household survey
that captures detailed information on food expenditure, maternal
education, and living standards. Although the APIS lacks anthropometric
data and has more limited geographic coverage, it enables exploratory
analyses of treatment effect heterogeneity by income and education
strata and helps triangulate changes in maternal resource constraints
following the policy shock.

Finally, missing data are handled following standard practices in
applied microeconometrics. Observations with missing or flagged
anthropometric outcomes are excluded from the analysis. For other
covariates with partial nonresponse---such as maternal BMI, anemia, or
household infrastructure---I adopt the missing-indicator method: missing
values are imputed with a constant, and a binary indicator is included
in all regression models to control for the fact of missingness. This
approach preserves sample size while allowing the data to inform whether
nonresponse is systematically related to the outcomes of interest. As
robustness checks, I estimate alternative specifications using multiple
imputation and inverse-probability weighting to test the sensitivity of
results to different missing data assumptions. All descriptive and
inferential statistics are computed using DHS-provided sampling weights,
and models account for the complex survey design by clustering standard
errors at the primary sampling unit.

\bibliographystyle{aea}
\bibliography{references}

\% The appendix command is issued once, prior to all appendices, if any.
\appendix

\section{Mathematical Appendix}


\end{document}
