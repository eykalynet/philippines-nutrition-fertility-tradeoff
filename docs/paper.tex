% AER-Article.tex for AEA last revised 22 June 2011
\documentclass[]{AEA}

% The mathtime package uses a Times font instead of Computer Modern.
% Uncomment the line below if you wish to use the mathtime package:
%\usepackage[cmbold]{mathtime}
% Note that miktex, by default, configures the mathtime package to use commercial fonts
% which you may not have. If you would like to use mathtime but you are seeing error
% messages about missing fonts (mtex.pfb, mtsy.pfb, or rmtmi.pfb) then please see
% the technical support document at http://www.aeaweb.org/templates/technical_support.pdf
% for instructions on fixing this problem.

% Note: you may use either harvard or natbib (but not both) to provide a wider
% variety of citation commands than latex supports natively. See below.

% Uncomment the next line to use the natbib package with bibtex
\usepackage{natbib}

% Uncomment the next line to use the harvard package with bibtex
%\usepackage[abbr]{harvard}

% This command determines the leading (vertical space between lines) in draft mode
% with 1.5 corresponding to "double" spacing.
\draftSpacing{1.5}


% tightlist command for lists without linebreak
\providecommand{\tightlist}{%
  \setlength{\itemsep}{0pt}\setlength{\parskip}{0pt}}


% Add imagehandling



\usepackage{pdfpages}
\usepackage{float}
\usepackage{makecell}
\setlength{\parskip}{0pt}       % Remove vertical space between paragraphs
\setlength{\parindent}{15pt}    % Indent first line of each paragraph
\usepackage{booktabs}
\usepackage{longtable}
\usepackage{array}
\usepackage{multirow}
\usepackage{wrapfig}
\usepackage{float}
\usepackage{colortbl}
\usepackage{pdflscape}
\usepackage{tabu}
\usepackage{threeparttable}
\usepackage{threeparttablex}
\usepackage[normalem]{ulem}
\usepackage{makecell}
\usepackage{xcolor}
\usepackage{siunitx}

    \newcolumntype{d}{S[
      table-align-text-before=false,
      table-align-text-after=false,
      input-symbols={-,\*+()}
    ]}
  

\usepackage{hyperref}

\begin{document}


\title{Who Gets Cared For and Why? Fertility Shocks, Intra-Household
Investment Reallocation, and the Quantity--Quality Trade-Off in the
Philippines}
\shortTitle{Spillovers from Policy-Induced Fertility}
% \author{Author1 and Author2\thanks{Surname1: affiliation1, address1, email1.
% Surname2: affiliation2, address2, email2. Acknowledgements}}


\author{
  Erika Salvador\\
  Caroline Theoharides\thanks{
  Salvador: Amherst
College, \href{mailto:esalvador28@amherst.edu}{esalvador28@amherst.edu}.
  Theoharides: Amherst
College, \href{mailto:ctheoharides@amherst.edu}{ctheoharides@amherst.edu}.
  This research was made possible through the support of the Schupf
  Fellowship at Amherst College. I am deeply thankful to Dr.~Axel Schupf
  '57 for his generosity and vision in establishing this program, which
  annually supports a cohort of twenty rising sophomores and juniors in
  independent summer research. I feel truly privileged to have been
  selected through this competitive process and to have spent eight
  weeks immersed in a project that both challenged and inspired me. I am
  especially grateful to my mentor, Prof.~Caroline Theoharides, for her
  mentorship throughout this project---and more broadly, for sparking my
  interest in economic research and its potential to serve the public
  good in the Philippines. I also wish to thank Faculty Director
  Professor Amelie Hastie and the many campus partners whose
  institutional support made this experience possible. Finally, I am
  grateful to the Departments of Economics and Mathematics \& Statistics
  for equipping me with the analytical and conceptual tools that made
  this research possible---and for preparing me to take on independent
  research with confidence and curiosity. All views and any errors are
  my own.
}
}

\date{\today}
\pubMonth{08}
\pubYear{2025}
\pubVolume{}
\pubIssue{}
\JEL{J13, I15, O15}
\Keywords{fertility shocks, child nutrition, quantity-quality
trade-off, urban poverty, Philippines}

\begin{abstract}
This paper investigates how increases in fertility, driven by reduced
access to contraception, affect parental investment in child health. I
use Executive Order No.~003---a 2000 policy in Manila that restricted
access to modern contraceptives---as an exogenous fertility shock.
Drawing on pooled data from the 1998, 2003, and 2008 Philippine
Demographic and Health Surveys, I estimate two-stage least squares
models that use policy exposure to instrument for sibship size. Child
health inputs fall into two categories: prenatal (proxied by birth
weight) and postnatal (measured by the total number of vaccine doses
received by age five). The first-stage results indicate that affected
mothers had, on average, 0.20 more living children at the time of a
birth. In the second stage, each additional sibling is associated with a
reduction of 22.97 vaccine doses received by the index child, a decline
large enough to prevent completion of the recommended immunization
schedule. The estimated effect on birth weight is −8,676 grams, though
statistically insignificant. Reduced-form estimates are directionally
consistent: policy exposure is associated with a 0.91-dose drop in
vaccine uptake and a 349.60-gram decrease in birth weight. While
instrumental variable estimates are limited by a weak first stage, the
magnitude and sign of the coefficients support the hypothesis that
larger sibship sizes reduce health investments per child. The results
suggest that postnatal investments, particularly those requiring
sustained parental effort, are more sensitive to fertility shocks than
prenatal outcomes.
\end{abstract}


\maketitle

\section{Introduction}

The trade-off between child quantity and child quality is a foundational
concept in the economics of the family. First articulated by
\citet{becker1960economic} and extended in subsequent models of
household behavior \citep{becker1973interaction, becker1976child}, this
framework posits that parents allocate finite resources---both financial
and non-financial---across children. An increase in fertility dilutes
the resources available per child and, under binding constraints, may
lead to reduced investments in each child's human capital.

Empirical investigations of the quantity-quality trade-off have focused
primarily on educational outcomes. Most studies examine how increases in
maternal fertility---or equivalently, the number of siblings a child
has---affect school enrollment, grade progression, test scores, and
completed years of schooling
\citep{rosenzweig1980testing, black2005more, angrist2010multiple}. These
outcomes serve as accessible proxies for long-run human capital
accumulation, but they represent only one dimension of child quality.
Other dimensions of child quality--such as health at birth and early
childhood health investments---are early-onset or biologically
constrained, and they tend to be less amenable to remediation later in
life. For instance, babies born with very low birth weight face higher
risks of infant mortality and long-term developmental deficits that
cannot be fully reversed \citep{cook2015understanding}. Similarly, a
child who misses critical early vaccinations is more vulnerable to
infectious diseases that can kill or disable them, which results to
lifelong negative health impacts \citep{costa2024child}. Such early-life
health setbacks may permanently impair cognitive development and
physical growth. This limits long-run productivity regardless of
subsequent improvements in schooling or income.

The omission of these health outcomes from much of the empirical
literature leaves important aspects of the quantity-quality trade-off
unexamined. Investments in maternal and child health---such as adequate
prenatal care leading to healthy birth weight and timely childhood
immunizations---are important to early development and long-term
outcomes \citep{victora2008maternal, hoddinott2013adult}. They shape
brain development, immune system functioning, and physical stature, and
they have been shown to predict later-life earnings and health outcomes
across a wide range of settings
\citep{grantham2007development, alderman2006long}. Unlike educational
gaps, which may be partially addressed later in life, early health
deficits---such as poor nutrition or missed vaccinations---are often
irreversible \citep{victora2008maternal, hoddinott2013adult}. As a
result, any credible evaluation of the quantity-quality trade-off must
include early-life health outcomes to fully capture the long-term
consequences of different fertility shocks.

This study addresses that gap by examining how a localized, exogenous
increase in fertility influenced early-life health outcomes in the
Philippines. In 2000, then--Mayor José ``Lito'' Atienza of Manila City
issued Executive Order No.~003, which effectively barred the provision
and promotion of modern contraceptives in all city-run health
facilities. The directive eliminated access to pills, condoms,
intrauterine devices, and other forms of modern family planning, and
mandated that public healthcare providers offer only natural methods.
The policy remained in effect for nearly a decade and applied solely
within the jurisdiction of the Manila city government. No comparable
restriction was implemented at the national level, and neighboring
cities within Metro Manila continued to provide access to modern
contraceptives. This policy created a clear spatial and temporal
discontinuity in access to family planning services---one that was
plausibly unrelated to changes in fertility preferences, household
income, or broader governance trends. As such, the Manila contraceptive
ban provides a quasi-experimental shock to fertility exposure among poor
urban households.

I use this natural experiment to estimate the causal impact of increased
family size on early-childhood health outcomes. The analysis uses
nationally representative data from multiple rounds of the Philippine
Demographic and Health Survey (DHS), which include detailed information
on household structure, maternal characteristics, and child health
indicators. The primary outcomes are birth weight (measured in grams,
when reported) and whether the child received key early-life
vaccinations, including those against polio,
diphtheria--pertussis--tetanus (DPT), and measles, as a measure of
essential postnatal health investment. These outcomes are widely
recognized indicators of child well-being and reflect both biological
and behavioral dimensions of parental investment in early development.
The empirical strategy follows a difference-in-differences design that
compares child outcomes in Manila and comparable urban areas before and
after the onset of the policy.

The identification strategy rests on two key assumptions. First, in the
absence of the contraceptive ban, trends in child health outcomes in
Manila would have evolved in parallel with those in the comparison
cities. Second, no other policy changes or economic shocks in Manila
during the study period coincided with the timing and scope of the
contraceptive policy. I probe these assumptions using falsification
tests, placebo comparisons, and robustness checks that include
city-specific time trends, mother fixed effects, and controls for
baseline demographic differences. The analysis proceeds in three stages.
I first replicate existing work \citep{dumas2019sex} to confirm that the
contraceptive ban led to an increase in fertility among affected women.
Next, I estimate the reduced-form effects of exposure to the policy on
the selected child health outcomes (birth weight, and vaccination
uptake). Finally, I examine heterogeneity in these effects across
subgroups defined by maternal education, household wealth, and access to
prenatal care. These characteristics proxy for a household's resources
and its capacity to buffer the impact of increased family size on child
health.

This study contributes to the literature in several important ways. It
provides new evidence on how an externally induced increase in family
size, resulting from a local restriction on contraceptive access, can
affect child health in poor urban communities. Whereas most past
research on the quantity-quality trade-off has focused on education,
this study broadens the notion of child ``quality'' to include
early-life health measures such as birth weight and immunization status.
It also adds to the small number of studies that leverage unexpected
changes in reproductive health policy to examine long-term effects on
children's well-being. More broadly, the findings indicate that local
restrictions on family planning can unintentionally harm children's
health, especially in settings where families already face poverty, food
insecurity, and limited public health services.

\section{Review of Related Literature}

The quantity-quality (Q--Q) theory, a central idea in modern family
economics, holds that parents face a trade-off between the number of
children and the ``quality'' of investment---such as education or
health---they can provide to each. Quality in this context refers to the
human capital of each child: attributes like education, health, and
nutrition that enhance a child's future productivity and well-being. The
genesis of this idea traces back to Gary Becker's seminal work around
1960, which for the first time treated children as economic goods
subject to parental choice and budget constraints
\citep{becker1960economic}. Becker argued that as families become
wealthier, they may not simply want more children, but rather
better-raised children, much as a household might prefer a
higher-quality car or house over a greater quantity of them. This
proposition led to a formal theory in which increases in income or
changes in economic conditions cause parents to substitute child quality
for quantity, consistent with historical patterns of lower fertility and
higher educational attainment during economic development
\citep{galor2000population}.

In what follows, I review the theoretical foundations of the Q-Q model
and its evolution in the literature. I begin with the static models of
Becker \citep{becker1960economic} and Becker--Lewis
\citep{becker1973interaction}, which first formalized the trade-off
within a household utility maximization framework. I then examine
extensions to dynamic, intergenerational settings, including the
contributions of Becker and Tomes \citep{becker1976child} on child
endowments and the altruistic dynastic model associated with Barro and
Becker \citep{barro1989fertility}. Next, I turn to macroeconomic and
unified growth models, notably Galor and Weil
\citep{galor2000population} and Galor and Moav \citep{galor2002natural},
which integrate the Q--Q mechanism into a general theory of demographic
and economic transformation.

Finally, I discuss more recent refinements that enrich the basic model
by incorporating credit constraints \citep{doepke2004accounting},
intra-household bargaining \citep{doepke2019bargaining}, and
multi-dimensional child quality
\citep{hoddinott2013economic, kalemli2002does}, with a special emphasis
on health and nutrition. The literature review focuses on how the Q--Q
framework has been applied to understand fertility and child investment
patterns, especially in developing country contexts where resource
constraints and health outcomes are paramount.

\subsection{Theoretical Background}

\subsubsection{Becker’s Static Model}

Becker's early work introduced an economic model of fertility and
treated children as durable goods that provide utility to parents but
impose costs \citep{doepke2015gary}. In Becker's 1960 model, a household
derives satisfaction from the number of children (\(n\)) and from the
quality of each child (\(q\)), alongside conventional consumption of
other goods (\(y\)). A simple representation is a utility function:

\[
U = U(n, q, y),
\]

with \(U\) increasing in each argument up to some satiation point. Here,
quality \(q\) can be thought of as the expenditure or investment per
child (e.g.~education spending, health care, nutrition), assumed for now
to be the same for each child. Parents face a budget constraint that
links quantity and quality: raising more children dilutes the resources
available per child. A prototypical budget constraint (in static form)
can be written as:

\[
p_y y + p_n n + p_q n q = I,
\]

where \(I\) is total family income (or full income), \(p_n\) represents
baseline, non-discretionary costs associated with each additional child
(e.g.~expenditures on food, shelter, or clothing that are incurred
irrespective of quality-enhancing investments), and \(p_q\) denotes the
marginal cost of investing in one unit of quality per child.The term
\(p_q n q\) captures total expenditure on quality for all children and
is linear in \(n\). As the number of children rises, parents must extend
any chosen level of \(q\) across a broader base, which amplifies the
total cost of quality. Conversely, the term \(p_n n\) implies that the
cost of an additional child rises with the quality level \(q\) already
chosen, since each child must meet a higher standard of care or
investment. For instance, a household that chooses to provide more
education or better health care per child incurs an additional burden
when it expands family size, as each child must receive the same
enhanced level of investment. Similarly, a larger family increases the
cumulative cost of quality, even if \(q\) remains fixed, due to the need
to replicate expenditures across more children. In short, the shadow
price of child quality increases with \(n\), and the shadow price of
child quantity increases with \(q\). The cost structure induces a mutual
dependence between quantity and quality, such that any adjustment along
one dimension alters the effective cost of the other.

Becker and Lewis (1973) formalize the mutually reinforcing nature of the
quantity-quality cost structure. An increase in \(n\) raises the total
cost required to sustain a given level of \(q\) for each child, while a
higher level of \(q\) raises the marginal cost associated with having an
additional child. For example, allocating more resources to education or
health per child increases the financial burden of expanding family
size. This interdependence links the two decisions directly. The
household cannot choose \(n\) and \(q\) in isolation; each choice alters
the marginal cost of the other.

Mathematically, the trade-off appears in the first-order conditions of
the household's optimization problem. Let \(\lambda\) represent the
Lagrange multiplier on the full-income constraint.

\[
\mathcal{L} = U(n, q, y) + \lambda \left( I - p_y y - p_n n - p_q n q \right).
\]

The first-order conditions are:

\[
\frac{\partial \mathcal{L}}{\partial n} = U_n - \lambda(p_n + p_q q) = 0, \quad
\frac{\partial \mathcal{L}}{\partial q} = U_q - \lambda p_q n = 0, \quad
\frac{\partial \mathcal{L}}{\partial y} = U_y - \lambda p_y = 0.
\]

Combining the first two yields:

\[
\frac{U_n}{U_q} = \frac{p_n + p_q q}{p_q n}.
\] This condition equates the marginal rate of substitution between
quantity and quality to the ratio of their full marginal costs. The
numerator rises with \(q\), and the denominator rises with \(n\). As one
choice increases, the relative cost of the other becomes higher. This
relationship induces substitution toward the less costly dimension. The
trade-off between quantity and quality arises from the structure of the
budget itself. It does not rely on specific assumptions about utility
curvature or intrinsic substitutability \citep{becker1973interaction}.

This formulation implies two core predictions. Firstly, although both
child quantity and child quality may rise with income, the household's
budget constraint can generate a negative relationship between income
and fertility. As income increases, total spending on children tends to
rise, but the allocation often favors quality over quantity. Becker
illustrated this with the analogy of durable goods: wealthier households
tend to upgrade the quality of a house or a car rather than acquire
additional units. In a similar way, higher-income families often direct
additional resources toward education, nutrition, or health per child.
Within the model, an income increase (\(\mathrm{d}I > 0\)) produces a
direct effect that makes children more affordable and an indirect effect
that discourages fertility. As \(q\) rises, the shadow price of an
additional child also rises. If the marginal utility from higher quality
exceeds that from larger family size, then the substitution effect
outweighs the income effect, leading to a lower optimal \(n\). This
mechanism offers a structural explanation for the demographic
transition: fertility tends to fall as households become richer, even
when preferences remain unchanged.

Furthermore, a similar logic applies to changes in the cost parameters
\(p_q\) and \(p_n\). A decline in \(p_q\), such as through a policy that
lowers the price of education or health care, increases \(q\) and raises
the marginal cost of quantity. This effect reduces optimal fertility. A
rise in \(p_n\), which may reflect higher child-rearing costs or a
greater opportunity cost of parental time, reduces the appeal of larger
families and can shift resources toward child quality. These outcomes
follow from the structure of the budget constraint, without requiring
any explicit preference for quality over quantity. Becker and Lewis
noted that these comparative static results align with observed
patterns. For example, increases in women's wages often reduce fertility
more than they reduce educational spending per child. This asymmetry
reflects the model's central feature: quantity and quality are linked
through their cost structure. An increase in one raises the marginal
cost of the other. The model explains how households make trade-offs
between the number of children and investments in each.

\subsubsection{Intergenerational Models: Altruism and Child Endowments}

While the early Q--Q models were static (one-period) representations,
subsequent contributions extended the framework to consider fertility
and child investment over multiple periods or even multiple generations.
The main development in this literature was the incorporation of
intergenerational human capital dynamics, where parents derive utility
not only from the number and quality of children in the present, but
also from the long-run outcomes of their offspring. These extensions
allowed child quality to evolve endogenously across time, rather than
being determined solely within a single period.

One of the earliest and most influential models of this kind was
proposed by \citet{becker1976child}. In their formulation, each child
enters the world with an exogenous endowment \(E\), which may reflect
factors such as cognitive ability (e.g., measured IQ or language
acquisition speed), early health status (e.g., birth weight or incidence
of neonatal complications), genetic predispositions (e.g., risk for
chronic illness, temperament, or neurodevelopmental traits), or family
background characteristics (e.g., parental education, household
stability, or neighborhood conditions). Furthermore, parents can augment
this endowment by investing resources \(q\) in the form of education,
nutrition, and other quality-enhancing inputs. The effective adult human
capital of the child might be expressed as \(H = E + f(q)\) (in a simple
additive form) or a multiplicative variant \(H = E \cdot f(q)\), where
\(f(q)\) is an increasing concave function of parental investment.

\citet{becker1976child} emphasized that variation in endowments \(E\)
can shape how parents allocate investments \(q\) across children. When
the productivity of investment increases with endowment, parents may
concentrate resources on children with higher \(E\), who are more likely
to convert additional investment into future success. In other cases,
parents may attempt to compensate for lower endowments by directing
greater investment toward disadvantaged children. Put simply,
child-specific variation in initial conditions affects not only outcomes
but also strategic parental choices. Because of this, the relationship
between income and the demand for child quality is not uniform. The
income elasticity of demand for quality may differ across households,
depending on the distribution of endowments within the family and across
the broader population.

Furthermore, \citet{becker1976child} showed that at low income levels,
much of what constitutes child quality comes from exogenous endowments,
i.e., factors like public education, neighborhood environment, or access
to basic healthcare that are not privately purchased. In these settings,
small increases in parental income may not lead to significant changes
in fertility or investment behavior. Since most of the child's future
outcomes are determined by the fixed endowment component, marginal
investment plays a smaller role. However, as income rises, private
resources become a larger part of what determines quality, and the
classic Q--Q trade-off begins to shape behavior. Parents begin to
allocate more income toward fewer children in order to enhance quality
through direct investment. Under certain theoretical conditions, such as
equal utility elasticities for quantity and quality, this framework
produces a non-monotonic relationship between income and fertility.
Fertility may decline as income rises at first, which reflects the
desire to invest more intensively per child. However, beyond some point,
once the marginal return to investment begins to flatten or saturate
relative to the fixed endowment, fertility may increase again.

The U-shaped prediction emerges only under specific assumptions, and its
validity depends on both the shape of the utility function and how
endowments relate to parental background. More broadly,
\citet{becker1976child} enriched the Q--Q framework by incorporating
elements that reflect real-world variation. They argued that not all
differences in child outcomes are the result of deliberate parental
choice. Random factors, biological traits, and socioeconomic settings
play a role. In this light, public policies, such as subsidized
schooling, early childhood programs, or universal healthcare, can
influence private fertility and investment decisions by shifting the
effective value of \(E\) across the population. If government programs
raise the floor for child endowments, then even low-income parents can
achieve better outcomes without large private sacrifices. These
policy-induced shifts in \(E\) alter the perceived return to having more
children or investing more per child.

\citet{becker1976child} also considered the possibility that endowment
is not randomly assigned but may vary systematically with income.
Higher-income households may produce children with higher \(E\) due to
better maternal nutrition, access to prenatal care, lower exposure to
environmental risk, or assortative matching on traits associated with
educational or occupational success. In these families, not only are the
resources available for investment greater, but the potential gains from
investment may also be higher, because children are better positioned to
benefit from those inputs. This interaction deepens the divide between
high- and low-income households, making it harder for disadvantaged
families to catch up. As a result, inequality can persist or even widen
across generations.

Finally, the model provides a mechanism for understanding how
imperfections in credit markets can lead to persistent disadvantages. If
parents with low income and low-\(E\) children cannot borrow to finance
quality-enhancing investment, then the next generation begins life with
the same disadvantage. Without external intervention or structural
change, this loop continues, which results in a pattern where poor
families remain poor and rich families accumulate further advantage. The
Becker-Tomes framework thus connects household-level decisions to bigger
questions about the intergenerational transmission of human capital.

Parallel to Becker and Tomes's static analysis of endowments, another
strand of the literature developed a fully dynamic version of the Q-Q
model by incorporating parental altruism toward children's welfare. In
this framework, introduced by \citet{barro1989fertility}, parents care
not only about the number and quality of their children but also about
the utility their descendants will enjoy in the future. Altruism in this
context means that parents treat their children's utility as part of
their own, thus extending the household's objective across generations.
For example, a parent may reduce personal consumption to pay for a
child's schooling, motivated not just by the child's immediate benefit
but by the satisfaction the parent gains from the child's long-term
success. This leads to a formulation of dynastic utility, where the
household's objective spans infinitely many periods and takes the form
of a recursive altruistic structure. A representative formulation is

\[
U_{0} = \sum_{t=0}^{\infty} \beta^{t}\,u(c_{t}, n_{t}),
\]

where \(c_t\) denotes the consumption of the \(t\)-th generation,
\(n_t\) the number of children, and \(\beta \in (0,1)\) the
intertemporal discount factor. Given this structure, having an extra
child \(n_t\) enters utility positively, but each child is assumed to
receive the same utility as the parent if raised at a comparable
standard of living. As a result, parents confront an intertemporal
trade-off: having more children expands the number of future utility
streams but also stretches current resources, since each child requires
support. This trade-off gives rise to an Euler equation for optimal
fertility choice, analogous to an optimal growth condition.

An implication of the dynastic model is that fertility decisions are
sensitive to macroeconomic conditions, such as the interest rate or the
rate of return on capital. A rise in interest rates increases the
opportunity cost of channeling resources into children rather than
saving, which tends to reduce current fertility---a substitution effect
across generations. At the same time, higher returns make future
generations wealthier, and this anticipated prosperity enters the
utility calculations of parents in more complex ways.
\citet{barro1989fertility} demonstrated that the model can account for
observed fertility responses to economic fluctuations and policy
interventions It can also explain historical phenomena such as postwar
baby booms and subsequent fertility declines through shifts in returns
or labor‐market opportunities.

In many dynastic models, child quality appears indirectly, often through
the child's future human capital or income. One variant assumes parents
value the aggregate human‐capital stock of their children. This
specification, combined with altruism, produces a similar trade‐off:
concentrating resources in fewer children raises each child's human
capital, which raises the dynasty's long‐run welfare. These
intergenerational extensions link micro‐level fertility decisions to
macroeconomic outcomes. By the late 1980s, work by Becker, Barro, and
others had recast fertility as an endogenous choice that interacts with
capital accumulation, income distribution, and policy. This laid the
foundation for unified growth theories, which view the quantity-quality
mechanism as central to demographic transition and long-run development.

\subsubsection{Unified Growth Models}

The unified growth theory, developed in the late 1990s and 2000s
(notably by Oded Galor and co-authors), seeks to explain in one
framework the entire sweep of economic development -- from Malthusian
stagnation, through the demographic transition, to modern growth. A
central puzzle it addresses is why fertility rates, which were
historically high and invariant to income in the Malthusian era, began
to decline sharply in tandem with industrialization and rising incomes,
eventually stabilizing at much lower levels in developed economies. The
Q--Q trade-off provides a key part of the answer in these models.
\citet{galor2000population} and \citet{galor2002natural} explicitly
incorporate parental choices about the quantity and quality of children
and show how changes in the economic environment alter those choices and
trigger demographic transitions.

In \citet{galor2000population}'s model, for instance, technological
progress gradually increases the return to human capital, especially in
skilled occupations. In the early stages, when production relies on
basic tools and techniques, unskilled labor holds more value. Under
these conditions, parents have little reason to invest in formal
schooling. Children are expected to contribute economically through
agricultural work, domestic tasks, or low-skill jobs in workshops and
factories. Fertility remains high because children impose a low
financial burden and generate immediate returns. As technology becomes
more advanced---such as during the Industrial Revolution---the earnings
gap between skilled and unskilled labor widens. Education begins to
offer significant advantages in the labor market. In response, parents
adjust by having fewer children and placing greater emphasis on each
child's development, including school attendance and better health care.

Evidently, industrialization raises the economic value of skilled labor,
which alters household incentives. As returns to education increase,
parents begin to favor investments in child quality over child quantity.
This shift results in declining fertility because families choose to
have fewer children and allocate more resources to each. The feedback
effect is significant: higher educational investment raises productivity
in the next generation, which in turn accelerates technological
advancement and further increases the returns to human capital. Over
time, the economy moves from a state of high fertility and low growth to
one characterized by low fertility and sustained growth.

Furthermore, \citet{galor2002natural} introduced an evolutionary
refinement to the unified growth framework by accounting for
heterogeneity in parental preferences. During the Malthusian period,
some families placed greater emphasis on child quality, such as
education, while others prioritized quantity. In a stagnant economy with
limited returns to education, high-fertility lineages maintained a
numerical advantage and suppressed average human capital. As
technological progress increased the returns to education, families that
valued quality gained an economic edge. Their children acquired more
human capital and achieved higher income and survival rates. These
advantages allowed such families to grow in relative size. Over time,
this process resembled a form of evolutionary selection, gradually
favoring quality-oriented parental types and shifting the population
toward greater emphasis on child human capital. These dynamics
strengthened the shift from high-fertility, low-education regimes to
low-fertility, high-investment family structures. In formal
overlapping-generations models, \citet{galor2002natural} demonstrate
that this evolutionary adaptation accelerates the demographic
transition. Their framework accounts for the rapid and widespread drop
in fertility once it takes effect. Higher returns to human capital push
parents to favor quality, while preferences for quality begin to
dominate within the population. These forces support the emergence of a
low-fertility, high-investment equilibrium and establish a unified
explanation for both economic development and demographic change.

Importantly, unified growth models identify several complementary
mechanisms that reinforce the basic Q--Q trade-off during development.
One is the decline of child labor. As the economy modernizes, the value
of child labor falls, both because legal reforms often restrict child
labor and because parents realize the earnings their children could make
as unskilled laborers are paltry compared to the potential returns if
those children instead spend time in school. \citet{hazan2002child}
formally show that when child labor becomes less profitable relative to
adult (skilled) labor, parents further reduce fertility and invest more
in each child's education. Historical evidence from England, for
instance, indicates that during industrialization the wages of children
(relative to adults) dropped significantly, especially in skilled
families, and this was accompanied by parents pulling children out of
work to send them to school. \citet{galor2006human} even argue that
capitalist industrialists supported public education laws and child
labor bans as a way to increase the human capital of the workforce,
inadvertently hastening the fertility transition.

Another mechanism is the rise in life expectancy and child survival.
Improvements in sanitation, nutrition, and medical knowledge in
developing societies led to more children surviving to adulthood. While
the earliest unified growth models treated mortality as exogenous or
ignored it, later research demonstrated that declining child mortality
can trigger lower fertility as well-- parents no longer need ``extra''
births for insurance once they are confident their existing children
will survive. In other words, increased child survival and the
quality--quantity trade-off are complementary explanations for fertility
decline that operate in tandem. When fewer births are lost to disease,
parents can achieve a desired number of surviving offspring with fewer
total births, and they tend to reallocate effort into each child's
health and education.

The overall effect is a reinforcing cycle: better health raises the
returns to schooling (healthier children can learn more effectively and
have longer working lives), which further encourages educational
investments and reduces fertility. Indeed, Galor notes that human
capital should be interpreted broadly to include health as well as
schooling; in unified growth theory, improvements in nutrition and
physical well-being were crucial to making labor more productive and
thus were part and parcel of the rise in demand for human capital.

The unified growth literature places the Q--Q model within a more
general account of economic and demographic change. In this framework,
higher income or stronger returns to child quality reduce fertility and
help shift economies from stagnation toward sustained growth. Several
mechanisms support this transition, such as a fall in child labor, a
drop in child mortality, and a shift in parental priorities. These
models explain not only the presence of a quantity-quality trade-off but
also its rising influence at a specific point in history. The evidence
supports these claims: countries that saw earlier increases in returns
to education experienced earlier fertility decline, while delays in
reforms, such as public education or health access, corresponded to
prolonged high fertility. As a result, the Q--Q mechanism forms a key
component of unified growth theory.

\subsection{Recent Refinements to the Q-Q Model}

Contemporary research has further refined the quantity-quality model by
relaxing some of its initial simplifying assumptions. Three important
extensions involve (1) capital market imperfections that constrain
parents' ability to invest in child quality, (2) intra-household
conflict and bargaining between mothers and fathers over fertility
choices, and (3) recognition that child quality is multi-dimensional,
which extends beyond schooling to include health, nutrition, and other
facets of human capital.

To begin, I examine how credit constraints can give rise to poverty
traps. The canonical quantity-quality (Q--Q) framework assumes that
parents can reallocate resources freely; borrowing against future
earnings to finance schooling or health investments whenever the
expected return is high. In practice, especially in low-income settings,
credit markets function imperfectly: poor households typically cannot
secure loans to cover children's education or medical costs even when
such investments would yield substantial future gains. This market
failure magnifies the Q--Q trade-off. Becker, Lewis, and Willis--already
noted by \citet{grawe2008quality} as emphasizing ``resource
limitations''--implicitly recognized this issue, but contemporary models
make it explicit by imposing a borrowing constraint. Parents must fund
childrearing from current income alone; they cannot collateralize a
child's future wages to pay today's school fees. Consequently, when
income is low, each additional birth directly reduces the attainable
quality per child, potentially trapping families in a
low-income--high-fertility equilibrium.

Formal models show that ``the quality-quantity trade-off arises from a
binding credit constraint that prevents parents from borrowing against
future child income.'' Empirical work supports this mechanism.
\citet{kremer2002income} and \citet{de2003inequality} document that
countries facing tighter liquidity constraints tend to display higher
fertility and lower educational attainment, consistent with
liquidity-constrained parents favoring quantity over quality.
Cross-country evidence likewise indicates that where financial frictions
are more severe, the negative correlation between fertility and
schooling is stronger. Theoretically, introducing a borrowing limit can
generate multiple steady states: one with low fertility and high
investment when incomes suffice to cover quality costs, and another with
high fertility and minimal investment when they do not. Policy
instruments such as education subsidies or conditional cash transfers
effectively relax these constraints, which nudge households toward the
low-fertility, high-investment equilibrium. In short, incorporating
credit market imperfections deepens the explanatory power of the Q-Q
model: economic growth alone may not reduce fertility if households
remain too cash-poor to afford schooling, whereas targeted
quality-enhancing transfers can catalyze both demographic and
human-capital transitions.

A further refinement of the quantity-quality (Q--Q) model considers the
question of \emph{who} within the household makes fertility and child
investment decisions. The original Beckerian framework adopts a unitary
model of the family, and assumes a single utility function and complete
agreement between spouses over optimal fertility \(n\) and child quality
\(q\). In practice, however, empirical evidence reveals significant
heterogeneity in preferences between household members along gender
lines. According to \citet{oppenheim1987impact, thomas1990intra}, for
instance, men often desire more children than women and may differ in
their willingness to invest in each child's education or health. These
discrepancies have motivated game-theoretic models of intra-household
bargaining, in which fertility and investment outcomes reflect the
relative influence of each parent's preferences.

In such models, the mother is typically assumed to have a stronger
preference for child quality---such as health and schooling---while the
father may favor either more children or alternative uses of household
resources. The resolution of these conflicting preferences depends on
bargaining power, which can be shaped by income contributions, legal
rights, cultural norms, or access to external resources. When the
mother's bargaining power increases, theoretical models predict a shift
toward lower fertility and higher per-child investment, holding other
factors constant. This prediction is consistent with empirical findings:
\citet{iyigun2007endogenous, doepke2019bargaining} show that greater
female empowerment---via education or labor force
participation---correlates with reduced fertility and increased
investment in child human capital.

Mathematically, the household's first-order condition for fertility in a
bargaining model can be written as:

\[
\alpha \frac{\partial U_{\text{wife}}}{\partial n} + (1 - \alpha) \frac{\partial U_{\text{husband}}}{\partial n} = \lambda(p_n + p_q q),
\]

with a corresponding condition for \(q\). Here, \(\alpha \in [0,1]\)
represents the wife's bargaining weight. When \(\alpha\) increases, the
composite marginal utility of additional children typically
decreases---especially if the wife prefers fewer children---leading to
lower equilibrium fertility and a shift along the Q--Q frontier toward
higher quality.

Recent work also explores dynamic bargaining, in which spouses negotiate
sequential decisions over time, potentially leading to strategic
behavior (e.g., one partner may accelerate or delay subsequent births).
Although these models introduce complexity, their core implication for
the Q--Q framework is clear: household power dynamics fundamentally
shape the trade-off between child quantity and quality. In societies
where women have limited decision-making autonomy---due to lack of
access to contraception, or social norms---fertility tends to remain
high and per-child investment low, which stalls demographic transition.
Conversely, when women gain bargaining power---through legal reforms,
labor market participation, or targeted transfers---the household often
reallocates resources toward fewer but higher-quality children.

This theoretical insight is corroborated by policy experiments. For
instance, cash transfer programs directed to mothers consistently lead
to greater spending on children's health, education, and nutrition
compared to equivalent transfers given to fathers
\citep{duflo2003grandmothers, thomas1990intra} . Such outcomes support
the hypothesis that mothers place higher weight on child quality, and
that who controls the purse strings matters deeply. In sum, introducing
intra-household bargaining into the Q--Q model enriches its explanatory
scope: it highlights how family structure and power asymmetries---not
just income levels or prices---generate variation in fertility and human
capital outcomes across and within societies.

\subsection{Health and Nutrition in the Q–Q Model}

The original Q--Q models often used a single catch-all variable for
child quality, typically thought of as education or ``expenditure per
child.'' Recent work emphasizes that child quality is multi-faceted, and
that parents make trade-offs along several dimensions of
investment---cognitive development, health, nutrition, etc. This is
salient in developing countries, where basic health and nutrition are
pressing concerns alongside schooling.

The theoretical question is how these dimensions interact with the
quantity decision. If parents allocate a budget across, say, schooling
\(q_{\text{edu}}\) and nutrition/health \(q_{\text{health}}\) for each
child, then having more children forces cutbacks in both dimensions
(unless parents reallocate across them). In some models, health and
education are complementary: a healthier child benefits more from
education, and an educated mother might raise a healthier child. This
complementarity can amplify the Q--Q trade-off---investing in one
dimension (health) increases the returns to investing in the other
(education), so a high-quality strategy becomes more focused on fewer
children.

On the other hand, if one dimension has diminishing returns more quickly
than another, parents might prioritize achieving a threshold level of
health for all children before adding more education, which creates a
nonlinear effect on fertility. One especially important aspect of health
in the Q--Q framework is child survival. The probability that a child
survives to adulthood effectively multiplies the utility of having that
child. Historically, high child mortality led to a strategy of
``quantity for insurance,'' where parents had additional births to
ensure survivorship. As mortality falls due to public health
improvements, parents can shift toward quality without risking
childlessness \citep{kalemli2002does, kalemli2000mortality}.

\citet{kalemli2002does} developed a stochastic model in which fertility
choices are made under uncertainty about child survival. Her results
show that declining mortality causally reduces fertility and increases
educational investment per child, in line with the Q--Q trade-off. In
unified growth models, declining child mortality reinforces the
demand-for-human-capital channel of the demographic transition
\citep{galor2004physical}.

Beyond survival, early-life health indicators such as birth weight and
vaccination status are important dimensions of child quality, especially
in low-income settings. When resources are limited, parents may struggle
to meet the basic needs of each child. Empirical evidence aligns with
the quantity--quality framework: children from larger families often
have worse health outcomes at birth and are less likely to receive core
immunizations. For example, recent multi-country data show that the
share of completely unimmunized children---so-called ``zero-dose''
children---increases from 10.5\% among those with no siblings to 17.2\%
among those with four or more \citep{costa2024child}. Studies from
Southeast Asia also report that higher birth order correlates with lower
average birth weight. These patterns suggest that as fertility
increases, the health of each child may suffer due to more limited
parental investment. Studies from Southeast Asia
\citep{chen2021population} also report that higher birth order
correlates with lower average birth weight.

Similar patterns emerge in sub-Saharan Africa
\citep{bishwakarma2019first} and South Asia, where first-born children
experience slower growth once younger siblings arrive---again pointing
to intrahousehold trade-offs. Theoretically, one can extend the Becker
model with a nutrition production function, where a child's health \(H\)
depends on food \(F\) and medical care \(M\), such that:

\[
H = g(F, M)
\]

and \(H\) enters either utility directly or the child's future
productivity. Parents then choose \(n\), \(F\), \(M\), and possibly
schooling per child. Larger \(n\) reduces \(F\) and \(M\) per child,
lowering \(H\); if parents place high value on \(H\), or if \(H\)
enhances returns to education, they will opt for smaller families.

Importantly, improvements in nutrition (through income growth or public
programs) can first lead to both better health and higher fertility
(since healthier women can bear more children). But over time, as
standards of living rise, better nutrition increases the returns to
investing in fewer, healthier children. Economic historian
\citet{fogel1994economic} argues that Europe's demographic transition
was partly driven by better diets: improved nutrition raised the
productivity of educated workers, which made education more valuable and
shifted family preferences toward quality.

These propositions are important for modern development policy. Family
planning programs that encourage lower fertility without improving child
health and education may have limited long-term effects. Conversely,
health and nutrition interventions---such as vaccinations or food
supplementation---can enhance the returns to education and induce
parents to reduce fertility voluntarily. The extended Q--Q framework
thus serves as a unifying tool to understand how diverse
interventions---ranging from school fees to nutrition programs---shape
long-run development trajectories.

\subsection{Philippine Evidence on the Q–Q Trade-off}

Early empirical work in the Philippines provided suggestive support for
this theory. For example, an influential study by
\citet{horton1986child} used Philippine household data to jointly
examine fertility and child nutrition and treated nutritional status as
a measure of child quality. Horton found that better-educated mothers
and fathers tended to choose smaller families and achieved
better-nourished children--evidence of substitution away from
``quantity'' toward child ``quality.'' Notably, she observed strong
birth-order effects on nutrition (first-born children faring better than
later-born), which hint that parents may not evenly distribute quality
investments among all children. This early work indicated a
quantity-quality (Q--Q) trade-off in Philippine families, though it
largely documented correlations rather than definitive causation.

Subsequent studies in the Philippines have tackled the critical issue of
causality, using innovative research designs to isolate exogenous
changes in family size. One seminal contribution is
\citet{orbeta2010number}, who examined the impact of family size on
children's schooling using a nationally representative survey.
Recognizing that family size is endogenous (parents' fertility choices
may reflect unobserved preferences or constraints), Orbeta employed an
instrumental-variable (IV) approach grounded in Becker's framework.
Specifically, he used the sex composition of the first two children as
an instrument for having additional children -- an approach pioneered by
\citet{angrist1996children} -- leveraging the fact that Filipino parents
often desire a mixed-gender sibset (and are more likely to have a third
child if the first two are the same sex) \citep{vicerra2013fertility}.
This strategy aims to generate random-like variation in family size
uncorrelated with parental characteristics.

The IV estimates confirmed a significant negative causal effect of
higher fertility on educational outcomes. \citet{orbeta2010number} found
that each additional child in the household reduced the proportion of
school-age children (6--24 years) attending school by roughly 19\% of
the baseline attendance rate. The trade-off was especially pronounced at
higher education levels: for example, the estimated drop in school
attendance was about 26\% at the secondary level and 57\% at the
tertiary level for each additional sibling. These are sizable effects,
implying that children from large families are substantially less likely
to remain in school, presumably due to tighter household budget
constraints or diluted parental attention. Moreover, the burden of the
trade-off appeared regressive: Orbeta's results showed much larger
schooling deficits from an extra child in poorer households than in
richer ones. For instance, in the poorest quintile, an additional
sibling reduced school attendance by an estimated 24\% (for ages 6--24),
compared to a 16\% reduction in the richest quintile. This regressive
pattern aligns with Becker's theory that resource constraints bind more
tightly for low-income families, which makes the Q--Q trade-off more
acute. In summary, Orbeta's study -- the first in the Philippines to
account for fertility endogeneity -- provides robust evidence that
increases in family size cause significant declines in child educational
attainment and validates the quantity-quality trade-off in this context.
Large family size thus emerges as one mechanism contributing to poverty,
by impeding children's human capital accumulation in the Philippines.

Further compelling evidence comes from a natural experiment studied by
\citet{dumas2019sex}. They exploit a unique policy shock in metropolitan
Manila to identify the trade-off. In 1998, the Mayor of Manila city
imposed a sudden ban on modern contraceptives in public facilities,
drastically curtailing access to family planning for residents of Manila
city (but not in surrounding municipalities). Dumas and Lefranc use this
policy as a quasi-experiment: comparing families in Manila city (treated
by the ban) to similar families in other cities unaffected by the ban
before and after 1998. This difference-in-differences design, coupled
with the fact that older mothers were naturally less fecund during the
ban, isolates an exogenous fertility increase.

The results are striking. The contraceptives ban led to a significant
rise in births and family size in Manila city relative to the control
areas. Correspondingly, children born in Manila during the ban era
experienced a sizable decline in educational attainment -- clear
evidence of a Q--Q trade-off precipitated by the shock. In the authors'
words, the policy-driven increase in family size ``provide{[}s{]}
evidence of a quality--quantity trade-off'': larger families, forced by
the ban, resulted in lower schooling outcomes per child.

Beyond education, researchers have also examined health and other child
outcomes in relation to family size. Evidence generally suggests the
Q--Q trade-off extends to child health and nutrition. For instance,
\citet{horton1986child} already hinted that large families may
compromise child nutrition for later-born siblings. More recent regional
research resonates with this. \citet{hatton2018fertility} analyze the
effect of fertility on child height (a long-run health indicator) using
longitudinal data from Indonesia -- a neighboring Southeast Asian
country with comparable developmental challenges. They address
endogeneity by exploiting Indonesia's family planning program rollout
and exposure to mass media as instruments for fertility. The authors
find a significant negative impact of family size on child health: each
additional sibling is associated with about a one-third standard
deviation reduction in a child's height-for-age Z-score, after
controlling for other factors. This health penalty from having more
children is strongest in low-education households and appears in both
urban and rural areas. Such findings mirror the Philippine evidence that
the harms of large family size are most pronounced among disadvantaged
families. In economic terms, poorer parents with many children struggle
to provide adequate nutrition and schooling to all, which highlight the
equity dimension of the Q--Q trade-off.

\subsection{Family Size and Child Outcomes in Southeast Asia}

The Philippine experience is echoed in other ASEAN countries, where
researchers have probed the quantity-quality trade-off with diverse
outcomes and methods. In Vietnam, for example, rapid fertility decline
alongside rising education led to questions about a Q--Q mechanism.
\citet{anh1998family} documented a negative correlation between family
size and children's school enrollment in Vietnam, though their analysis
could not fully establish causality.

More rigorously, \citet{dang2016decision} used distance to the nearest
family planning center as an instrument to study Vietnamese households'
investments in education. They introduced a novel measure of child
quality---spending on private tutoring, a prevalent form of educational
investment in Vietnam---alongside traditional indicators like schooling
expenses. The IV estimates confirmed that children with more siblings
receive significantly lower educational investments from their families.
In particular, Vietnamese families of larger size spent less on each
school-age child's schooling and tutoring, even after controlling for
community factors. This effect was robust across different definitions
of family size and model specifications, indicating that Vietnamese
parents do indeed trade off quantity for quality when faced with
resource constraints. Such evidence aligns squarely with Becker's model:
as Vietnamese family size increases, per-child education spending falls;
parents are prioritizing ``quality'' less when they have more offspring.

Indonesia and Thailand show similar patterns. Demographic research in
Thailand during its fertility transition found that large families had
markedly worse educational outcomes. \citet{knodel1990family}, studying
Thai data in the 1990s, observed that once family size exceeded about
4--5 children, the likelihood of a child progressing to or staying in
secondary school dropped precipitously compared to smaller families.
Although these early Thai studies were based on correlations, they
strongly suggested that limited family resources were being spread thin
in big families and hurt children's schooling attainment.

In Indonesia, cohort analyses and natural experiments reinforce the
trade-off. \citet{maralani2008changing} showed that the relationship
between sibship size and schooling evolved from neutral or even positive
for older cohorts (born when education opportunities were limited) to
negative for more recent cohorts, consistent with a growing importance
of education in a modernizing economy. More concretely, the
aforementioned study by \citet{hatton2018fertility} in Indonesia
provides causal evidence that mirrors the Philippine findings in health
and nutrition. Likewise, an analysis of Indonesian census data
\citep{feng2021effect} found that having additional siblings
significantly lowers children's educational attainment once birth order
effects are accounted for, paralleling results from China and Vietnam.

These regional studies share a commonality: in resource-constrained
settings of Southeast Asia, increased child quantity tends to come at
the expense of child quality, be it years of schooling, academic
spending, or health status. The consistency of this pattern---across
countries with different cultures and policies---highlights the
fundamental economic logic identified by \citet{becker1973interaction}.
Parents with finite resources face difficult choices, and many appear to
balance quantity and quality in a way that confirms the trade-off
hypothesis.

Despite these similarities, there are some noteworthy nuances and gaps
in the ASEAN literature. One is the role of public policy and
development level. Evidence suggests that the Q--Q trade-off may be
mitigated in contexts with strong public support for education and
health. For instance, studies in developed countries (e.g.~Israel,
Norway) often find little or no trade-off once factors like birth order
are accounted for
\citep{black2005more, kristensen2010educational, angrist2010multiple}.
In Southeast Asia, however, public education quality and social safety
nets are still developing, and the cost of raising children (education
fees, food, etc.) is largely borne by families themselves
\citep{oecd2024sigi}. This may explain why the trade-off emerges so
clearly in the Philippines, Vietnam, Indonesia, and Thailand.

Another nuance is methodological: more recent studies employ credible
identification strategies (IVs, twins, policy shocks) and consistently
find a causal negative effect of family size on child outcomes, whereas
older studies without such controls sometimes found weaker effects or
none at all. This highlights the importance of accounting for
endogeneity.

Finally, there remain gaps for future research. Most ASEAN studies focus
on education and early-life health indicators as measures of child
quality; there is relatively little evidence on long-term outcomes such
as children's eventual earnings or income in adulthood. It is not yet
fully clear whether the schooling and health disadvantages observed in
larger families translate into significantly lower adult productivity or
income---a link that \citet{becker1976child} theorized but which could
be explored further in this region.

Additionally, while the trade-off appears pervasive, its magnitude can
vary: for example, the penalty of an extra child may be larger in poorer
rural areas than in urban or wealthier settings, suggesting that local
context (poverty, gender norms, access to services) can modulate the
trade-off. Comparative studies across ASEAN are still somewhat limited,
and a common challenge is disentangling related factors like birth
order, sibling composition, and parental preferences.

Nonetheless, the prevailing evidence from the Philippines and its
regional neighbors strongly supports Becker's quantity-quality
conjecture. As families have fewer children, they appear to invest more
in each child's education---investments crucial for human capital
development and economic growth. Conversely, high-fertility households
risk under-investing per child, which reinforces cycles of poverty and
inequality. This literature thus provides an important empirical
foundation for policies in the Philippines and ASEAN---from family
planning programs to education subsidies---that aim to ease the
quantity-quality trade-off and help families achieve both manageable
size and better outcomes for the next generation.

\section{Conceptual Framework and Hypotheses}

This study builds upon the foundational framework of the
quantity-quality (Q--Q) trade-off, first articulated in
\citet{becker1960economic} and later refined in
\citet{becker1973interaction}. In this model, fertility and child
quality are jointly determined under a full-income constraint, such that
increases in the number of children---without a corresponding expansion
in household resources---necessitate a reduction in the average
investment per child. While empirical applications of this framework
have traditionally focused on educational outcomes, child quality can
also be expressed in terms of early-life health, including indicators
that emerge at birth or shortly after, such as birth weight, and uptake
of essential vaccinations.

This analysis focuses specifically on early-life health as a proxy for
child quality. These outcomes, unlike cognitive or schooling-related
investments, are biologically constrained and occur within a narrow
developmental window, during which parental inputs are least
substitutable and most consequential. The trade-off is expected to be
especially salient in settings where fertility increases exogenously
while family resources remain fixed. In such contexts, a sudden increase
in sibship size may compel households to spread food, healthcare, and
caregiving attention across more dependents, which would potentially
lead to diminished health investments and outcomes for each child.

To formalize this relationship, consider a household with total income
\(I\), choosing nutritional investments \(\kappa_j\) for each of \(n\)
children, along with consumption \(Z\) unrelated to child quality:

\[
I = \sum_{j=1}^{n} \kappa_j + Z.
\]

In the presence of a fertility shock, holding \(I\) constant, the
average \(\kappa_j\) must fall unless offset by changes in parental
behavior or external transfers. If these inputs decline and are critical
to early-life development, observable deficits may arise in birth
weight, and children may be less likely to receive complete immunization
schedules.

The first hypothesis derived from this framework is that an exogenous
increase in fertility leads to a decline in per-child health investment,
particularly in low-income urban households with little financial slack.
A second, related hypothesis posits that this decline results in
measurable early-life health disadvantages---specifically, lower birth
weight and reduced uptake of key childhood vaccinations. These
hypotheses describe a causal pathway through which fertility shocks may
contribute to early deprivation and potentially reinforce
intergenerational cycles of poor health and limited human capital.

\section{Data and Empirical Strategy}

\subsection{Data}

This study draws on two primary data sources: the Philippine Demographic
and Health Surveys (DHS) and the Integrated Public Use Microdata Series
(IPUMS) census data. The DHS provides detailed child- and
household-level information on fertility behavior, maternal
characteristics, and early childhood health outcomes. I use three rounds
of the DHS---1998, 2003, and 2008---which are the only Philippine DHS
waves with publicly available GPS cluster data. These data are used to
classify respondents into treatment and control groups. Because
city-level identifiers are not included in the public-use DHS files, I
rely on approximate geographic assignment using cluster coordinates.
Clusters located within or near the administrative boundaries of the
City of Manila are classified as treated, while clusters located
elsewhere in the National Capital Region (NCR) serve as the control
group.

The DHS analytic sample is limited to children under five years of age
who were living with their mothers at the time of the survey. I examine
three core early-life health outcomes: (1) reported birth weight (in
grams); (2) maternal perception of birth size; and (3) vaccination
uptake, measured as the receipt of key early childhood immunizations
(BCG, DPT, polio, and measles). These outcomes are consistently
available across all three survey rounds and are commonly used as
proxies for child quality in studies of the quantity--quality trade-off.

To complement the DHS analysis and support the identification strategy,
I construct a panel of individual-level microdata from the IPUMS
harmonized Philippine censuses (1990, 2000, and 2010). These nationally
representative surveys provide a snapshot of women of reproductive age
(15--49) and contain detailed fertility histories---including total
children ever born, births in the past year, and number of surviving
children. I limit the sample to women ages 15--49 and construct a
composite fertility index combining several measures: total children
ever born, births in the past year, number of children under age five,
and the age of the youngest child. By comparing Manila with other
provinces and municipalities in NCR, I test whether long-run fertility
patterns reflect the contraceptive policy shock.

The 1998 DHS serves as the primary pre-policy baseline, capturing births
prior to the issuance of Executive Order No.~003. The 2008 DHS is the
main post-policy round, while the 2003 DHS is treated as a transition
period. I exclude the 1993 DHS due to the absence of geospatial
identifiers. In contrast, the census years 1990 and 2000 bracket the
policy's onset, while the 2010 census provides post-policy fertility
outcomes. This combination of sources enables both short-run and
long-run assessments of the policy's impact on fertility and child
investment.

To validate the policy shock, I use DHS region- and city-level
contraceptive use indicators, which confirm a sharp and sustained
decline in modern contraceptive use in Manila relative to other NCR
cities. These patterns reinforce the plausibility of Executive Order
No.~003 as an exogenous shock to fertility preferences and access. I
also considered the Philippine National Nutrition Survey (NNS) as an
auxiliary source, but its limited pre-2003 coverage restricts its
utility for identifying pre-policy trends. For this reason, the combined
DHS and IPUMS datasets offer the most comprehensive and temporally
balanced data available for the study's objectives.

\subsection{Variables}

The primary outcome variables reflect early-life health and investment.
These include: (1) birth weight in grams, as recorded by the mother when
available; and (2) indicators for whether the child received core
early-life vaccinations, including BCG, DPT1--3, polio doses, and
measles.

The main explanatory variable is sibship size, defined as the total
number of surviving children ever born to the mother at the time of the
child's birth, as reported in the birth history module. This variable
captures the resource dilution mechanism central to the
quantity--quality trade-off. To isolate exogenous variation in sibship
size, I construct an instrument for fertility based on policy exposure:
an interaction of a Manila indicator and a post-ban cohort indicator.
Specifically, children born in the City of Manila in or after 2000 are
coded as treated. In extended specifications, I interact this treatment
indicator with maternal age at birth to allow for heterogeneity in
fertility responses across age cohorts, considering the fact that older
women likely exhibited lower responsiveness due to reduced fecundity.

Control variables include standard child-, maternal-, and
household-level characteristics. At the child level, I control for sex,
birth order, multiple birth status, and the preceding birth interval in
months. Maternal covariates include age at birth, educational attainment
(none, primary, secondary, higher), literacy, employment status, and age
at first birth. Household-level controls include household size,
urban/rural residence (restricted to NCR in the main analysis), and the
DHS wealth index. All regressions include survey year fixed effects to
control for secular trends and common shocks. Estimates are weighted
using DHS sample weights, and standard errors are clustered at the
primary sampling unit (PSU) level.

\subsection{Estimation}

To estimate the causal effect of fertility on early-life health
outcomes, I implement a two-stage least squares (2SLS) strategy. The
first stage is estimated at the maternal level, as fertility is a
cumulative outcome of the mother rather than the child. Importantly,
this stage also serves to establish whether the contraceptive supply ban
had a measurable impact on maternal fertility behavior. Specifically, I
use two measures of fertility: the total number of children ever born
and the number of living children at the time of survey. These outcomes
capture distinct dimensions of fertility---cumulative births and net
surviving offspring---and are regressed on an indicator for policy
exposure, defined as the interaction between a Manila municipality
indicator and a post-ban birth cohort.

In the second stage, I shift to the child level and use the number of
surviving older siblings at the time of the index child's birth as the
endogenous regressor, instrumented by policy exposure. This variable
captures sibship size at birth, and allows me to estimate the effect of
increased family size---induced by the contraception ban---on early-life
health outcomes such as birth weight, and number of vaccines received.
This two-stage framework allows me to isolate plausibly exogenous
variation in family size induced by the contraceptive supply ban in
Manila, while addressing endogeneity concerns that arise when fertility
is influenced by unobserved preferences or constraints. I estimate all
models using linear probability models (LPMs) for consistency with the
2SLS framework. All specifications apply DHS sampling weights and
cluster standard errors at the PSU level.

\section{Results and Discussion}

\subsection{First-Stage Estimates: Effect of the Contraceptive Ban on Fertility}

To assess the fertility effects of Executive Order No.~003 (EO 003)---a
2000 Manila City ordinance that curtailed access to modern
contraceptives---I estimate first-stage ordinary least squares (OLS)
regressions that relate policy exposure to maternal fertility outcomes,
using data from the Demographic and Health Surveys (DHS). These
first-stage estimates serve as the foundation for the instrumental
variable strategy in the next stage of analysis. The regressions use the
mother as the unit of analysis, since fertility is a cumulative outcome
determined at the maternal level. Evidence that the policy affected
maternal fertility behavior provides a necessary condition for
interpreting later estimates of child health outcomes as causal effects
of fertility.

The regression specification is designed to isolate the differential
change in fertility among mothers residing in Manila following the
implementation of EO 003, relative to mothers in other regions over the
same period. I estimate the following equation:

\[
\text{Fertility}_{im} = \beta_0 + \beta_1 \text{Manila}_m + \beta_2 \text{ManilaPost}_{im} + \boldsymbol{X}_{im}'\gamma + \lambda_{y(i)} + \varepsilon_{im}
\]

Where: \(Fertility_{im}\) is the fertility outcome for mother \(i\) in
municipality m (e.g., number of living children at survey, total
children ever born); \(\text{Manila}_m\) is an indicator for residence
in Manila,\(\text{ManilaPost}_{im}\) is an interaction term equal to 1
if the mother resided in Manila and gave birth in or after the year 2000
(the policy period); \(X_{im}\) includes maternal characteristics such
as age, educational attainment, and household wealth; \(\lambda_{y(i)}\)
are birth year fixed effects, controlling for national trends in
fertility by cohort, and \(\varepsilon_{im}\) is the error term,
clustered at the primary sampling unit (PSU) level.

The regression results, shown in Table 4, examine the relationship
between policy exposure and sibship size. I use two measures of maternal
fertility: the number of living children at the time of survey (Model 1)
and the total number of children ever born (Model 2). In both
specifications, the coefficient on the interaction term Manila × 2000 is
positive---0.098 in Model 1 and 0.206 in Model 2---though neither is
statistically significant. These estimates suggest a directionally
consistent increase in fertility among Manila women exposed to the
contraceptive ban. Maternal education and household wealth are
negatively associated with fertility, while maternal age is positively
associated, consistent with socioeconomic gradients in childbearing.

Table \ref{tab:first-stage-sibship} presents first-stage regression
estimates assessing the relationship between exposure to Manila's
post-2000 fertility-restrictive policy and sibship size. The outcome
variables are the number of living children at the time of survey (Model
1) and the total number of children ever born, including deceased (Model
2). In both models, the interaction term Manila × Post-2000 is
positive---0.098 (SE = 0.691) in Model 1 and 0.242 (SE = 0.461) in Model
2---indicating that children born to Manila mothers after the policy may
have experienced slightly larger sibship sizes, on average. Although
these estimates are not statistically significant, their consistent
positive direction suggests that the policy did not produce a strong
fertility-decreasing effect, and may even have been associated with
modest increases in completed or surviving fertility.

\begin{table}[!h]
\centering\centering
\caption{\label{tab:first-stage-sibship}First-Stage Regressions: Effect of Policy Exposure on Sibship Size}
\centering
\resizebox{\ifdim\width>\linewidth\linewidth\else\width\fi}{!}{
\fontsize{10}{12}\selectfont
\begin{tabular}[t]{lcc}
\toprule
  & Model (1) & Model (2)\\
\midrule
Manila & \num{0.158} & \num{0.242}\\
 & (\num{0.333}) & (\num{0.142})\\
Manila × Post-2000 & \num{0.098} & \num{0.200}\\
 & (\num{0.691}) & (\num{0.461})\\
Mother’s Education & \num{-0.326}*** & \num{-0.397}***\\
 & (\num{<0.001}) & (\num{<0.001})\\
Mother’s Age & \num{0.200}*** & \num{0.221}***\\
 & (\num{<0.001}) & (\num{<0.001})\\
Wealth Index & \num{-0.250}*** & \num{-0.291}***\\
 & (\num{<0.001}) & (\num{<0.001})\\
\bottomrule
\multicolumn{3}{l}{\rule{0pt}{1em}+ p $<$ 0.1, * p $<$ 0.05, ** p $<$ 0.01, *** p $<$ 0.001}\\
\multicolumn{3}{l}{\rule{0pt}{1em}All regressions include birth year fixed effects.}\\
\multicolumn{3}{l}{\rule{0pt}{1em}Standard errors are clustered at the PSU level.}\\
\multicolumn{3}{l}{\rule{0pt}{1em}Estimates are weighted using DHS sample weights.}\\
\end{tabular}}
\end{table}

The control variables function as expected and lend credibility to the
model. Maternal education is significantly associated with smaller
sibship sizes (−0.326 and −0.397, p \textless{} 0.001), while maternal
age is positively associated (0.200 and 0.221, p \textless{} 0.001),
reflecting life-cycle fertility dynamics. Wealth index is also
negatively associated with fertility (−0.250 and −0.291, p \textless{}
0.001). Both models explain substantial variation in the outcome, with
adjusted R² values of 0.503 and 0.527, respectively.

Meanwhile, Table \ref{tab:first-stage-sibship-placebo} serves as a
placebo test using Quezon
City\footnote{Quezon City and Manila are both highly urbanized cities within Metro Manila and share closely aligned demographic profiles. Data from the 2010 Census and the 2013 NDHS show that both cities had similar rates of secondary or higher education among women (approximately 55--60\%), high institutional delivery coverage (above 90\%), and comparable access to modern contraceptive methods. Prior to the policy, total fertility rates in both cities ranged between 2.5 and 2.7 children per woman. These commonalities extend to urban infrastructure and wealth distribution, and make Quezon City a credible counterfactual for isolating the effects of Manila’s 2000 policy.\cite{psa2010,ndhs1998,ndhs2013,mmda2012}},
an urban area demographically and geographically comparable to Manila
but not subject to the policy. The placebo interaction term Quezon ×
Post-2000 is near zero in both specifications: 0.006 (SE = 0.951) for
living children and 0.036 (SE = 0.754) for total children ever born.
These findings are precisely null and much smaller in magnitude than
those observed for Manila, which provide important reassurance that the
patterns seen in Table \ref{tab:first-stage-sibship} suggest that the
observed effects are not confounded by region-wide demographic shifts or
by secular trends in urban fertility behavior. In contrast to Manila,
there is no directional indication of post-2000 fertility increase in
Quezon City.

\begin{table}[!h]
\centering\centering
\caption{\label{tab:first-stage-sibship-placebo}Placebo Regressions: Using Quezon City as Falsification Test}
\centering
\resizebox{\ifdim\width>\linewidth\linewidth\else\width\fi}{!}{
\begin{tabular}[t]{lcc}
\toprule
  & Model (1) & Model (2)\\
\midrule
Quezon City & \num{0.025} & \num{0.011}\\
 & (\num{0.750}) & (\num{0.904})\\
Quezon × Post-2000 & \num{0.006} & \num{0.036}\\
 & (\num{0.951}) & (\num{0.754})\\
Mother’s Education & \num{-0.325}*** & \num{-0.396}***\\
 & (\num{<0.001}) & (\num{<0.001})\\
Mother’s Age & \num{0.200}*** & \num{0.220}***\\
 & (\num{<0.001}) & (\num{<0.001})\\
Wealth Index & \num{-0.249}*** & \num{-0.289}***\\
 & (\num{<0.001}) & (\num{<0.001})\\
\bottomrule
\multicolumn{3}{l}{\rule{0pt}{1em}+ p $<$ 0.1, * p $<$ 0.05, ** p $<$ 0.01, *** p $<$ 0.001}\\
\multicolumn{3}{l}{\rule{0pt}{1em}Placebo test using Quezon City as a comparison group.}\\
\multicolumn{3}{l}{\rule{0pt}{1em}All regressions include birth year fixed effects.}\\
\multicolumn{3}{l}{\rule{0pt}{1em}Standard errors are clustered at the PSU level.}\\
\multicolumn{3}{l}{\rule{0pt}{1em}Estimates are weighted using DHS sample weights.}\\
\end{tabular}}
\end{table}

Table \ref{tab:first-stage-sibship-manila-qc} strengthens this
interpretation by directly comparing Manila to Quezon City in a
difference-in-differences framework. Restricting the sample to these two
cities, the coefficient on the Manila × Post-2000 term remains positive:
0.025 (SE = 0.925) in Model 1 and 0.104 (SE = 0.724) in Model 2.
Although these estimates are again statistically insignificant, they are
directionally aligned with the first-stage results and notably larger
than those in the placebo regressions. This suggests that even when
directly comparing policy-exposed and unexposed cities within a single
model, the post-2000 trend in Manila is somewhat more pronounced, and
offers support for a modest, policy-related fertility shift.

\begin{table}[!h]
\centering\centering
\caption{\label{tab:first-stage-sibship-manila-qc}Difference-in-Differences: Manila vs. Quezon City, Post-2000}
\centering
\resizebox{\ifdim\width>\linewidth\linewidth\else\width\fi}{!}{
\begin{tabular}[t]{lcc}
\toprule
  & Model (1) & Model (2)\\
\midrule
Manila & \num{0.116} & \num{0.204}\\
 & (\num{0.499}) & (\num{0.254})\\
Manila × Post-2000 & \num{0.025} & \num{0.104}\\
 & (\num{0.925}) & (\num{0.724})\\
Mother’s Education & \num{-0.274}*** & \num{-0.331}***\\
 & (\num{<0.001}) & (\num{<0.001})\\
Mother’s Age & \num{0.156}*** & \num{0.172}***\\
 & (\num{<0.001}) & (\num{<0.001})\\
Wealth Index & \num{-0.269}*** & \num{-0.350}***\\
 & (\num{<0.001}) & (\num{<0.001})\\
\bottomrule
\multicolumn{3}{l}{\rule{0pt}{1em}+ p $<$ 0.1, * p $<$ 0.05, ** p $<$ 0.01, *** p $<$ 0.001}\\
\multicolumn{3}{l}{\rule{0pt}{1em}Sample restricted to Manila and Quezon City.}\\
\multicolumn{3}{l}{\rule{0pt}{1em}All regressions include birth year fixed effects.}\\
\multicolumn{3}{l}{\rule{0pt}{1em}Standard errors are clustered at the PSU level.}\\
\multicolumn{3}{l}{\rule{0pt}{1em}Estimates are weighted using DHS sample weights.}\\
\end{tabular}}
\end{table}

To complement the regression analysis, I construct approximate birth
rates using four waves of the Philippine Census (1990, 1995, 2000, and
2010). I calculate city-level birth rates by year and compare trends
across Manila, other cities in the National Capital Region (NCR), and
the rest of the Philippines. Although these census-based estimates are
less precise than those derived from the DHS, they provide an
independent view of fertility patterns during the policy period. I
estimate annual birth rates by dividing the number of young children
(ages 0--5) in a given census year by the number of women of
childbearing age who could have plausibly given birth to them, based on
age and co-residence within the same household. To improve the
likelihood of correctly linking mothers and children, I restrict the
sample to women identified as household heads or spouses, under the
assumption that children typically reside with their biological mothers.

Formally, for each calendar year \(t\) and maternal age group \(a\), the
approximate birth rate is defined as:

\[
\text{BirthRate}_{a,t} =
\frac{
  \text{Number of children aged } c - t \text{ in census year } c
}{
  \text{Number of women aged } c - t + a \text{ in census year } c
}
\]

This approach estimates the share of women aged \(a\) in year \(t\) who
gave birth, inferred from the presence of children born in year \(t\)
(that is, children aged \(c - t\)) in the census conducted in year
\(c\). I use census years \(c \in {1990, 1995, 2000, 2010}\) and
construct birth rates based on children aged 0 to 5 at the time of
enumeration, thereby covering calendar years from 1985 to 2015. To
reduce noise and ensure comparability across time, I aggregate across
maternal ages 15 to 49.

Table \ref{tab:birthrates-region} compares average birth rates before
and after 2000 across three region groups: Manila, other NCR cities, and
the rest of the Philippines. The results reveal a notably smaller
decline in birth rates in Manila relative to the comparison areas.
Between the pre-2000 and post-2000 periods, Manila's average birth rate
decreased by only 8.4\% (from 0.0822 to 0.0753). In contrast, other NCR
cities experienced a 17.0\% decline, and the rest of the country saw a
20.8\% drop.

\begin{table}[!h]
\centering
\caption{\label{tab:birthrates-region}Average Birth Rates Before and After 2000 by Region Group}
\centering
\begin{tabular}[t]{lrrr}
\toprule
Region Group & Pre-2000 & Post-2000 & Percent Change\\
\midrule
Manila & 0.0822 & 0.0753 & -8.4\\
Other NCR Cities & 0.0920 & 0.0764 & -17.0\\
Rest of the Philippines & 0.1261 & 0.0999 & -20.8\\
\bottomrule
\end{tabular}
\end{table}

Across all regions, birth rates declined over time. However, the decline
was smallest in Manila (−8.4\%)---the only jurisdiction directly
affected by the contraceptive ban---compared to other NCR cities
(−17.0\%) and the rest of the Philippines (−20.8\%). These descriptive
trends suggest that although fertility fell nationwide, the reduction in
Manila was markedly smaller. This relative divergence aligns with the
regression-based findings, which pointed to a potential dampening of the
fertility decline in Manila following the implementation of EO 003. The
smaller post-2000 decline in birth rates supports the interpretation
that the policy may have altered fertility trajectories in ways not
observed elsewhere. If EO 003 influenced fertility behavior, we would
expect birth rates in Manila to decline more slowly than in unaffected
areas---and the data appear consistent with that expectation.

I recognize that any comparison involving the rest of the Philippines
must account for the sheer size and heterogeneity of this group, which
spans highly urbanized centers as well as rural and remote provinces.
Aggregating such diverse regions may obscure meaningful local variation
and exaggerate the contrast with Manila. That said, if Executive Order
No.~003 had a substantive effect on fertility behavior, I would still
expect fertility trends in Manila to diverge from those in the rest of
the country---a pattern that both the DHS regressions and census-based
birth rates consistently support.

In addition, several important caveats apply to the birth rate
approximation. First, the method omits both child and maternal
mortality, as well as any migration that may have occurred between the
time of birth and the census year. Second, the computation restricts the
sample to women who appear as household heads or as spouses of household
heads, to increase the likelihood that observed children are their
biological offspring. This restriction introduces composition effects.
For instance, a 26-year-old woman listed as a household head in the
census may differ systematically from a 25-year-old who assumed that
role several years earlier. These differences are especially salient
among younger women, where household formation often coincides with
marriage and initial childbearing. As a result, birth rates may show
artificial dips in the census years themselves.

Despite these limitations, the estimates help build confidence in the
plausibility of a fertility response to EO 003. They show that even when
fertility is measured approximately from census structure, the direction
of change aligns with expectations under a binding contraceptive ban.
This, in turn, motivates the more rigorous analysis that follows, which
uses DHS birth histories and econometric models to estimate fertility
effects more precisely.

\subsection{Reduced-Form Estimates: Effect of the Policy on Child Outcomes}

Before proceeding to instrumental variables (IV) estimation, I begin by
estimating reduced-form regressions to assess the intent-to-treat (ITT)
effects of policy exposure on child outcomes. This step serves two
purposes. First, it offers a preliminary test of whether the policy
shock---treated as exogenous variation in contraceptive access---leads
to measurable differences in child health. If no such differences
emerge, any IV estimates may lack credibility or relevance. Second, the
reduced-form regressions provide the denominator in two-stage least
squares (2SLS), which estimates the total effect of the instrument
without isolating its mechanism through an endogenous variable such as
fertility.

The reduced-form specification follows a difference-in-differences
framework. Formally, the reduced-form equation takes the following
structure:

\[
\text{Outcome}_{im} = \beta_0 + \beta_1 \text{Manila}_m + \beta_2 \text{ManilaPost}_{im} + \boldsymbol{X}'_{im} \gamma + \lambda_{y(i)} + \varepsilon_{im}
\]

Where: \(\text{Outcome}_{im}\) is the child health outcome for child
\(i\) of mother \(m\) (e.g., birth weight in grams or number of vaccine
doses received); \(\text{Manila}_m\) is an indicator for residence in
Manila; \(\text{ManilaPost}_{im}\) is an interaction term equal to 1 if
the child was born in Manila in or after the year 2001 (i.e., the policy
period), and 0 otherwise; \(\boldsymbol{X}'_{im}\) includes individual-
and household-level controls such as maternal age, maternal education,
household wealth index, birth order, child sex, and twin status;
\(\lambda_{y(i)}\) denotes birth year fixed effects that flexibly
capture national trends in child outcomes across cohorts; and
\(\varepsilon_{im}\) is the error term, clustered at the primary
sampling unit (PSU) level. The coefficient \(\beta_2\) captures the
reduced-form (intent-to-treat) effect of policy exposure on child
outcomes.

Table \ref{tab:reducedform-child-outcomes} reports the results from
reduced-form regressions estimating the intent-to-treat (ITT) effects of
EO 003 exposure on child health outcomes. Each column presents a
separate specification, with outcomes including birth weight, and number
of vaccines received

\begin{table}[!h]
\centering\centering
\caption{\label{tab:reduced-form-outcomes}Reduced-Form Regressions: Intent-to-Treat Effects of EO 003 on Child Outcomes}
\centering
\resizebox{\ifdim\width>\linewidth\linewidth\else\width\fi}{!}{
\begin{tabular}[t]{lcc}
\toprule
  & Birth Weight & Number of Vaccines\\
\midrule
Manila × Post-2000 & \num{-349.597} & \num{-0.910}***\\
 & (\num{0.142}) & (\num{<0.001})\\
Mother’s Age & \num{-35.581}*** & \num{0.048}***\\
 & (\num{<0.001}) & (\num{<0.001})\\
Mother’s Education & \num{-457.639}*** & \num{0.047}+\\
 & (\num{<0.001}) & (\num{0.061})\\
Wealth Index & \num{-549.139}*** & \num{-0.078}**\\
 & (\num{<0.001}) & (\num{0.005})\\
Birth Order & \num{182.550}*** & \num{-0.241}***\\
 & (\num{<0.001}) & (\num{<0.001})\\
Twin & \num{-397.679}* & \num{0.117}\\
 & (\num{0.016}) & (\num{0.489})\\
Child is Male & \num{-7.062} & \num{-0.051}\\
 & (\num{0.869}) & (\num{0.307})\\
\bottomrule
\multicolumn{3}{l}{\rule{0pt}{1em}+ p $<$ 0.1, * p $<$ 0.05, ** p $<$ 0.01, *** p $<$ 0.001}\\
\multicolumn{3}{l}{\rule{0pt}{1em}All regressions include birth year fixed effects.}\\
\multicolumn{3}{l}{\rule{0pt}{1em}Standard errors are clustered at the PSU level.}\\
\multicolumn{3}{l}{\rule{0pt}{1em}Estimates are weighted using DHS sample weights.}\\
\end{tabular}}
\end{table}

The strongest and most consistent finding emerges in vaccination uptake.
Exposure to the policy is associated with a significant reduction of
0.91 vaccine doses (p \textless{} 0.001). Given the Philippine
Department of Health's recommendation of eight core immunizations by age
one \citep{doh2023nip}, this decline represents a shortfall of over 10\%
of the complete vaccination schedule. A child born in Manila in
2001---after the implementation of Executive Order No.~003---was, on
average, likely to have missed nearly one routine vaccine compared to a
counterpart born just a few years earlier or outside Manila. This gap
could correspond to missing a third DPT booster, skipping the measles
vaccine, or failing to complete the polio series---each of which
increases exposure to otherwise preventable diseases.

In contrast, the estimated effect on birth weight is smaller and not
statistically significant. On average, children born under the policy
weighed 349.6 grams less than their peers in unaffected areas (p =
0.142). While the estimate does not reach statistical significance, its
magnitude is nontrivial: a shortfall of nearly 350 grams could shift a
child from normal birth weight into the low birth weight category
(\textless2,500g), which has been linked to higher risks of infant
morbidity and developmental delays \citep{katz2013mortality}.

Several covariates exhibit statistically significant associations with
early-life health outcomes. Maternal age shows a negative association
with birth weight (−35.6g per year, p \textless{} 0.001), which is
consistent with clinical evidence linking advanced maternal age to
elevated risks such as placental insufficiency and gestational
complications \citep{lean2023placental}. In contrast, each additional
year of maternal age corresponds to a modest increase in vaccine uptake
(+0.048 doses, p \textless{} 0.001), which might suggest greater
maternal experience or improved navigation of health systems
\citep{barker2021vaccination}. Meanwhile, maternal education shows a
split pattern. Children of more educated mothers weigh less at birth
(−457.6g, p \textless{} 0.001) but receive slightly more vaccines
(+0.047 doses, p = 0.061). These results may reflect reporting
differences: highly educated mothers may recall vaccination events more
precisely and assess birth size more conservatively, which result in
downward bias in reported weight \citep{filmer2001health}. Similarly,
wealth index also exhibits statistically significant associations,
though the direction diverges from conventional expectations. Higher
wealth correlates with lower birth weight (−549.1g, p \textless{} 0.001)
and fewer vaccine doses (−0.078, p = 0.005).

On the other hand, birth order has opposing effects. Later-born children
weigh more (+182.6g, p \textless{} 0.001) but receive fewer vaccines
(−0.241 doses, p \textless{} 0.001), which support theories of postnatal
resource dilution within larger families
\citep{black2008maternal}.Furthermore, twins weigh significantly less
than singletons (−397.7g, p = 0.016), which is consistent with evidence
of growth constraints in multiple gestations
\citep{klebanoff2005multiple}. However, twin status shows no
statistically significant effect on vaccination. Finally, child sex
carries no meaningful association with either outcome. Male children
weigh slightly less (−7.1g) and receive fewer vaccines (−0.051 doses),
but both estimates are small and statistically insignificant (p = 0.869
and 0.307), which suggest negligible gender disparities in early-life
health inputs in this context.

Taken together, these reduced-form estimates provide preliminary support
for a quantity--quality trade-off in early-life investments. However,
because exposure to the fertility shock (i.e., EO 003) is jointly
determined with family size, these estimates conflate the effects of
sibship size with the direct effects of policy exposure and local health
service constraints. To isolate the causal impact of family size per se,
I turn to an instrumental variable approach in the next section.

\subsection{Second-Stage Estimates: The Quantity–Quality Trade-off}

Having established preliminary intent-to-treat (ITT) effects of policy
exposure on child outcomes through reduced-form regressions, I now turn
to instrumental variables (IV) estimation to assess the causal effect of
sibling exposure on early childhood health. The IV approach allows for
identification of the total effect of having more siblings at birth on
outcomes such as birth weight and immunization, to address the potential
endogeneity of fertility decisions. Specifically, fertility may be
jointly determined with unobserved factors such as household
preferences, or income shocks, which would bias ordinary least squares
(OLS) estimates.

To address this concern, I instrument the number of living siblings at
birth using a policy-induced fertility shock: the interaction between
residence in Manila and the post-2000 period (i.e., after the issuance
of Executive Order No.~003). This interaction captures exogenous
variation in fertility arising from localized restrictions in
contraceptive access. The identifying assumption is that, conditional on
covariates and year fixed effects, this policy-driven variation affects
child outcomes only through its effect on sibling exposure.

The IV model is estimated using two-stage least squares (2SLS), where
the first stage regressed sibling exposure on the instrument and
controls, and the second stage regresses child outcomes on the predicted
sibling exposure. The estimating equation is of the following form:

\[
\text{Outcome}_{im} = \beta_0 + \beta_1 \widehat{\text{Siblings}}_{im} + \boldsymbol{X}'_{im}\gamma + \lambda_{y(i)} + \varepsilon_{im}
\]

Where \(\text{Outcome}_{im}\) is the child health outcome for child
\(i\) of mother \(m\); \(\widehat{\text{Siblings}}_{im}\) is the
instrumented number of living siblings at birth; \(\boldsymbol{X}_{im}\)
includes individual- and household-level controls (maternal age,
maternal education, wealth, birth order, sex, twin status); and
\(\lambda_{y(i)}\) denotes birth year fixed effects. Estimates are
weighted using DHS sample weights, and standard errors are clustered at
the primary sampling unit (PSU) level.

Table\textasciitilde{}\ref{tab:iv-child-outcomes} presents instrumental
variables (IV) estimates of the effect of sibling exposure---measured by
the number of living siblings at birth---on two key early childhood
outcomes: birth weight and vaccine uptake. These results represent an
important step toward isolating the causal impact of fertility changes,
building on the intent-to-treat estimates previously shown in Table 7.

\begin{table}[!h]
\centering\centering
\caption{\label{tab:second-stage-outcomes}Instrumental Variables Regressions: Effect of Sibling Exposure on Child Outcomes}
\centering
\resizebox{\ifdim\width>\linewidth\linewidth\else\width\fi}{!}{
\begin{tabular}[t]{lcc}
\toprule
  & Birth Weight & Number of Vaccines\\
\midrule
Living Siblings at Birth & \num{-8676.020} & \num{-22.968}\\
 & (\num{0.324}) & (\num{0.265})\\
Mother’s Age & \num{-209.721} & \num{-0.412}\\
 & (\num{0.236}) & (\num{0.319})\\
Mother’s Education & \num{-168.554} & \num{0.813}\\
 & (\num{0.568}) & (\num{0.240})\\
Wealth Index & \num{-754.195}*** & \num{-0.625}\\
 & (\num{<0.001}) & (\num{0.204})\\
Birth Order & \num{1089.466} & \num{2.155}\\
 & (\num{0.236}) & (\num{0.316})\\
Twin & \num{-1169.464} & \num{-1.887}\\
 & (\num{0.148}) & (\num{0.310})\\
Child is Male & \num{140.637} & \num{0.352}\\
 & (\num{0.407}) & (\num{0.386})\\
\bottomrule
\multicolumn{3}{l}{\rule{0pt}{1em}+ p $<$ 0.1, * p $<$ 0.05, ** p $<$ 0.01, *** p $<$ 0.001}\\
\multicolumn{3}{l}{\rule{0pt}{1em}Instrument: Manila × Post-2000}\\
\multicolumn{3}{l}{\rule{0pt}{1em}All regressions include birth year fixed effects.}\\
\multicolumn{3}{l}{\rule{0pt}{1em}Standard errors are clustered at the PSU level.}\\
\multicolumn{3}{l}{\rule{0pt}{1em}Estimates are weighted using DHS sample weights.}\\
\end{tabular}}
\end{table}

Although none of the coefficients on sibship size reach statistical
significance, the point estimates are large, negative, and consistent
with the quantity--quality trade-off hypothesis. Specifically, an
additional sibling is associated with an estimated 8,676-gram reduction
in birth weight and a decrease of 23 vaccine doses. These magnitudes,
while clearly inflated due to the wide standard errors and the weakness
of the instrument, nevertheless suggest that if fertility expands in
response to exogenous shocks, there may be meaningful consequences for
child health. The lack of precision does not negate the possibility of
real effects---it merely limits our ability to detect them definitively
in this sample. Importantly, the sign and direction of the effects
mirror those in the reduced-form estimates from Table 7, which
strengthens the plausibility of the underlying mechanism: policy-driven
increases in fertility may stretch household resources in ways that
reduce investments per child.

The control variables provide additional insight into household and
maternal characteristics that shape child outcomes. Birth order, for
instance, is positively associated with both birth weight and vaccine
uptake, which suggest that later-born children in this sample may
benefit from parental learning or behavioral adaptation, even in the
presence of additional siblings. In contrast, being a twin is associated
with over 1,100 grams less in birth weight---a biologically expected
result---while also linked to lower vaccination, which may reflect both
logistical barriers and prioritization decisions in constrained
households. The negative association between maternal age and birth
weight could reflect accumulated health risks or socioeconomic selection
among older mothers, while the small and imprecise effects of maternal
education and wealth on vaccination uptake indicate that formal
schooling and household resources alone do not guarantee equal access to
preventive care.

One surprising result is the strongly negative and statistically
significant association between the wealth index and birth weight (−754
grams, p \textless{} 0.001). This counterintuitive finding may reflect
residual confounding or differences in reporting accuracy, but it also
invites further investigation into whether urban wealth, where the cost
of living is higher and stress exposure more intense, might offset the
nutritional advantages typically associated with wealth in rural
settings. Taken together, the IV estimates do not offer causal
confirmation of the fertility--health linkage, but they suggest that the
hypothesized trade-off remains empirically relevant. The patterns
observed are coherent with theoretical expectations and underscore the
importance of future work using stronger instruments or alternative
policy shocks to credibly identify the full effect of family size on
child outcomes.

\section{Limitations and Future Work}

As with any empirical study, my analysis has important limitations.
First, although I use an instrumental variable approach to address
endogeneity in sibship size, the identification strategy rests on the
assumption that policy exposure influences child outcomes only through
its effect on fertility. While I believe this is a reasonable assumption
given the policy context, I am unable to fully rule out other
pathways---such as maternal stress or reduced postnatal care
access---that may also be at play.

Second, my analysis is constrained by the limitations of the Demographic
and Health Surveys (DHS). Key aspects of early childhood
development---such as stunting and wasting---are not available in the
DHS data I used, which restricts the scope of health outcomes I can
examine. This gap points to a promising direction for future work:
utilizing the Philippine National Nutrition Survey (NNS), which contains
more detailed anthropometric indicators. Similarly, while I measure
vaccination uptake by counting the number of doses received, I am unable
to determine whether those vaccines were administered on schedule or if
the child completed the full recommended series by age one. This
distinction is important because delays in vaccination can reduce
effectiveness and leave children vulnerable during critical
developmental windows. Since some surveys do record vaccine dates, I
hope to explore these timing dynamics more closely in future analyses.

Third, some of the outcomes I analyzed---especially perceived birth
size---are based on maternal reports, which may be shaped by cultural
expectations or subject to recall bias. Even birth weight, which is more
objective, is missing for a share of children, especially in earlier
survey waves or rural settings. This raises concerns about selection
bias because the children with available data may differ systematically
from those without. As I discussed earlier in the reduced-form results,
there's a notable disconnect between reported birth size and measured
birth weight. This discrepancy warrants closer attention---not just
because it may signal reporting bias, but also because it might reflect
how mothers perceive and evaluate infant health. Future work could
investigate whether these gaps differ systematically across groups, such
as by maternal education, geographic region, or socioeconomic status,
and what that might reveal about prevailing health knowledge and
beliefs.

Finally, the setting of Executive Order No.~003 in Manila limits how far
these results can be extended to other contexts. The policy was
geographically specific and politically shaped, with uneven
implementation across barangays and varying access to reproductive
services. \citet{dumas2019sex} noted that informal restrictions on
contraceptive access in Manila began as early as 1997, well before the
official issuance of Executive Order No.~003 in 2000. Even after its
formal implementation, the policy's enforcement was uneven---varying
significantly by barangay and often shaped by the discretion of local
health officials. These inconsistencies further complicate efforts to
isolate the policy's timing and intensity. These features make it
difficult to apply the findings to places without similar institutional
and demographic conditions. Although I include survey-year fixed
effects, the DHS data structure does not allow me to follow the same
clusters over time, which restricts my ability to test for pre-policy
trends or conduct spatially detailed robustness checks.

Future research can address these limitations by combining experimental
or quasi-experimental variation in fertility with longitudinal
administrative data on health. Replicating this framework in other
natural experiments---such as decentralization-driven contraceptive
rollouts or fertility shocks following natural disasters---would test
the robustness and generalizability of the quantity--quality trade-off.
In doing so, future work can help inform integrated policy approaches
that explain how families adjust their investments when fertility
changes.

\section{Conclusion}

This study investigates the impact of exogenous fertility shocks on
early childhood outcomes using the case of Executive Order No.~003 in
Manila, which imposed restrictions on access to modern contraceptives
beginning in 2000. Exploiting this localized policy change as a natural
experiment, I employ both reduced-form regressions and instrumental
variables (IV) estimation to assess whether increased sibling
exposure---proxied by the number of living siblings at birth---affects
child health investments and outcomes, specifically birth weight and
vaccine uptake.

The reduced-form results show that exposure to the policy is
significantly associated with lower vaccination coverage, and
directionally negative---though imprecise---for birth weight. These
findings suggest that the fertility increase induced by the policy may
have had measurable consequences for children's health inputs. The IV
estimates, while statistically insignificant due to weak instrument
strength, are consistent in sign and magnitude with the reduced-form
results. Point estimates suggest large negative effects of sibling
exposure on both outcomes, and provide suggestive---though not
definitive---evidence in support of the quantity--quality trade-off
framework. The consistency across models reinforces the possibility
that, under resource constraints, an exogenous rise in fertility may
reduce parental investments per child.

\bibliographystyle{aea}

\bibliography{references}


\end{document}
