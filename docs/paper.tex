% AER-Article.tex for AEA last revised 22 June 2011
\documentclass[]{AEA}

% The mathtime package uses a Times font instead of Computer Modern.
% Uncomment the line below if you wish to use the mathtime package:
%\usepackage[cmbold]{mathtime}
% Note that miktex, by default, configures the mathtime package to use commercial fonts
% which you may not have. If you would like to use mathtime but you are seeing error
% messages about missing fonts (mtex.pfb, mtsy.pfb, or rmtmi.pfb) then please see
% the technical support document at http://www.aeaweb.org/templates/technical_support.pdf
% for instructions on fixing this problem.

% Note: you may use either harvard or natbib (but not both) to provide a wider
% variety of citation commands than latex supports natively. See below.

% Uncomment the next line to use the natbib package with bibtex
\usepackage{natbib}

% Uncomment the next line to use the harvard package with bibtex
%\usepackage[abbr]{harvard}

% This command determines the leading (vertical space between lines) in draft mode
% with 1.5 corresponding to "double" spacing.
\draftSpacing{1.5}


% tightlist command for lists without linebreak
\providecommand{\tightlist}{%
  \setlength{\itemsep}{0pt}\setlength{\parskip}{0pt}}


% Add imagehandling




\usepackage{hyperref}

\begin{document}


\title{Nutritional Deficits and the Quantity-Quality Trade-off: Evidence
from an Exogenous Fertility Shock in Low-Income Urban Settings in the
Philippines}
\shortTitle{The Q-Q Trade-off in the Philippine Context}
% \author{Author1 and Author2\thanks{Surname1: affiliation1, address1, email1.
% Surname2: affiliation2, address2, email2. Acknowledgements}}


\author{
  Erika Salvador\\
  Caroline Theoharides\thanks{
  Salvador: Amherst
College, \href{mailto:esalvador28@amherst.edu}{esalvador28@amherst.edu}.
  Theoharides: Amherst
College, \href{mailto:ctheoharides@amherst.edu}{ctheoharides@amherst.edu}.
  This research was made possible by the Schupf Fellowship at Amherst
  College. I am indebted to my faculty adviser, Professor Caroline
  Theoharides, for her mentorship and support throughout the development
  of this project. I also wish to thank Faculty Director Professor
  Amelie Hastie and the campus partners whose institutional support
  enabled this work. I am further grateful to the Departments of
  Economics and Mathematics \& Statistics for their academic support.
  The views expressed and any errors contained herein are entirely my
  own responsibility.
}
}

\date{\today}
\pubMonth{06}
\pubYear{2025}
\pubVolume{}
\pubIssue{}
\JEL{J13, I15, O15}
\Keywords{fertility shocks, child nutrition, quantity-quality
trade-off, urban poverty, Philippines}

\begin{abstract}
This paper examines whether increased fertility affects early-life
nutritional outcomes in low-income urban households. I exploit a natural
experiment created by a 1990 policy in Manila, Philippines, which banned
modern contraceptives from city-run health facilities. Using a
difference-in-differences framework and nationally representative data
from the Philippine Demographic and Health Surveys, I estimate the
reduced-form impact of the policy on child height-for-age and
weight-for-height. {[}Will add more after data analysis{]}
\end{abstract}


\maketitle

\section{Introduction}

The trade-off between child quantity and child quality is a foundational
concept in the economics of the family. First articulated by
\citet{becker1960economic} and extended in subsequent models of
household behavior \citep{becker1973interaction, becker1976child}, this
framework posits that parents allocate finite resources---both financial
and non-financial---across children. An increase in fertility reduces
the resources available per child and, under binding constraints, may
lead to lower investments in health, education, and other forms of human
capital. This mechanism has served as an explanatory model for changes
in fertility behavior and the evolution of population structures in low-
and middle-income countries.

Empirical investigations of the quantity--quality trade-off have focused
primarily on educational outcomes. Studies in both high-income and
low-income settings have examined the effects of fertility on school
enrollment, grade progression, test scores, and completed years of
schooling
\citep{rosenzweig1980testing, black2005more, angrist2010effects}. These
outcomes serve as accessible proxies for long-run human capital
accumulation, but they represent only one dimension of child quality.
Other outcomes, such as nutritional status, are early-onset or
biologically constrained. They also tend to be less responsive to
remediation later in life. A child who suffers from chronic malnutrition
may exhibit permanently reduced cognitive capacity and face limits in
physical development that affect long-run productivity regardless of
subsequent educational access
\citep{hoddinott2013adult, grantham2007development}.

The exclusion of nutritional outcomes from much of the empirical
literature leaves an important dimension of the trade-off untested.
Nutritional investments in early childhood are essential to early
childhood development and long-term outcomes
\citep{victora2008maternal, hoddinott2013adult}. They shape brain
development, immune system functioning, and physical stature, and they
have been shown to predict later-life earnings and health outcomes
across a wide range of settings
\citep{grantham2007development, alderman2006long}. The biological
irreversibility of early-life nutritional deficits further distinguishes
them from other forms of investment. Educational deficits may be
partially remediable; nutritional failures often are not. A credible
estimate of the trade-off between fertility and child quality must
account for nutrition if it aims to assess the full set of consequences
associated with fertility shocks.

This study addresses this gap by examining the nutritional effects of a
localized, exogenous increase in fertility in the Philippines. In 1990,
the mayor of Manila implemented an executive policy that prohibited the
provision of modern contraceptives in all city-run health facilities.
The order removed access to pills, condoms, intrauterine devices, and
related public health materials and instructed healthcare providers to
offer only natural family planning methods. This policy remained in
place for nearly a decade and affected only the jurisdiction of the
Manila city government. The national government did not implement a
comparable restriction, and surrounding cities within Metro Manila
continued to provide access to modern contraceptives. The policy thus
created a spatial and temporal discontinuity in contraceptive access
that was uncorrelated with underlying fertility preferences or
concurrent shifts in household income or governance. As a result, the
Manila ban serves as a quasi-experimental source of variation in
fertility exposure among poor urban households.

I use this natural experiment to estimate the causal effect of increased
fertility on child nutrition. The analysis relies on nationally
representative data from multiple waves of the Philippine Demographic
and Health Survey (DHS), which provide data on household structure and
maternal characteristics, as well as measurements of child
anthropometry. The outcomes of interest are height-for-age and
weight-for-height z-scores, which serve as standardized indicators of
chronic and acute malnutrition, respectively. These outcomes are widely
used in the global health and development literature and capture
nutritional deprivation over both long and short time horizons
\citep{victora2008maternal}. The empirical strategy follows a
difference-in-differences design that compares child outcomes in Manila
and comparable urban areas before and after the onset of the policy.

The identification strategy rests on two key assumptions. First, in the
absence of the contraceptive ban, nutritional trends in Manila would
have evolved in parallel with those in comparison cities. Second, any
other policy or economic shocks affecting Manila during the study period
must not coincide precisely with the timing and scope of the
contraceptive policy. I test these assumptions using falsification
checks, placebo comparisons, and robustness specifications that include
city-specific time trends, maternal fixed effects, and controls for
baseline demographic differences.

The analysis proceeds in three stages. I first replicate existing work
\citep{dumas2019sex} to confirm that the contraceptive ban led to an
increase in fertility among affected women. I then estimate reduced-form
effects of policy exposure on nutritional outcomes for children under
five years of age. Finally, I examine heterogeneity in effects across
subsamples defined by maternal education, household wealth, and access
to prenatal care. These dimensions serve as proxies for household
resource availability and capacity to buffer the nutritional
consequences of fertility increases.

This study contributes to the literature in several important ways. It
provides new evidence on how increases in fertility---caused by policy
restrictions on family planning---can affect child nutrition in poor,
urban communities. Most past research has focused on education, but this
study expands the idea of child quality to include biological measures
such as stunting and wasting. It also adds to the small number of
studies that use unexpected changes in reproductive health policy to
examine long-term effects on children's well-being. More broadly, the
results show that local restrictions on family planning can
unintentionally harm children's health, especially in settings where
families already face poverty, food insecurity, and limited public
services.

\section{Review of Related Literature}

The quantity--quality (Q--Q) theory, a central idea in modern family
economics, holds that parents face a trade-off between the number of
children and the ``quality'' of investment---such as education or
health---they can provide to each. Quality in this context refers to the
human capital of each child: attributes like education, health, and
nutrition that enhance a child's future productivity and well-being. The
genesis of this idea traces back to Gary Becker's seminal work around
1960, which for the first time treated children as economic goods
subject to parental choice and budget constraints
\citep{becker1960economic}. Becker argued that as families become
wealthier, they may not simply want more children, but rather
better-raised children, much as a household might prefer a
higher-quality car or house over a greater quantity of them. This
proposition led to a formal theory in which increases in income or
changes in economic conditions cause parents to substitute child quality
for quantity, consistent with historical patterns of lower fertility and
higher educational attainment during economic development
\citep{galor2000population}.

In what follows, I review the theoretical foundations of the Q-Q model
and its evolution in the literature. I begin with the static models of
Becker \citep{becker1960economic} and Becker--Lewis
\citep{becker1973interaction}, which first formalized the trade-off
within a household utility maximization framework. We then examine
extensions to dynamic, intergenerational settings, including the
contributions of Becker and Tomes \citep{becker1976child} on child
endowments and the altruistic dynastic model associated with Barro and
Becker \citep{barro1989fertility}. Next, I turn to macroeconomic and
unified growth models, notably Galor and Weil
\citep{galor2000population} and Galor and Moav \citep{galor2002natural},
which integrate the Q--Q mechanism into a general theory of demographic
and economic transformation.

Finally, I discuss more recent refinements that enrich the basic model
by incorporating credit constraints \citep{doepke2004accounting},
intra-household bargaining \citep{doepke2019bargaining}, and
multi-dimensional child quality
\citep{hoddinott2013economic, kalemli2002does}, with a special emphasis
on health and nutrition. My focus is on how the Q--Q framework has been
applied to understand fertility and child investment patterns,
especially in developing country contexts where resource constraints and
health outcomes are paramount.

\subsection{Theoretical Background}

Becker's early work introduced an economic approach to fertility,
treating children analogous to ``durable goods'' that yield utility to
parents but come at a cost \citep{doepke2015gary}. In Becker's 1960
model, a household derives satisfaction from the number of children
(\(n\)) and from the quality of each child (\(q\)), alongside
conventional consumption of other goods (\(y\)). A simple representation
is a utility function:

\[
U = U(n, q, y),
\]

with \(U\) increasing in each argument up to some satiation point. Here
quality \(q\) can be thought of as the expenditure or investment per
child (e.g.~education spending, health care, nutrition), assumed for now
to be the same for each child. Parents face a budget constraint that
links quantity and quality: raising more children dilutes the resources
available per child. A prototypical budget constraint (in static form)
can be written as:

\[
p_y y + p_n n + p_q n q = I,
\]

where \(I\) is total family income (or full income), \(p_n\) represents
baseline, non-discretionary costs associated with each additional child
(e.g.~expenditures on food, shelter, or clothing that are incurred
irrespective of quality-enhancing investments), and \(p_q\) denotes the
marginal cost of investing in one unit of quality per child.The term
\(p_q n q\) captures total expenditure on quality for all children and
is linear in \(n\). As the number of children rises, parents must extend
any chosen level of \(q\) across a broader base, which amplifies the
total cost of quality. Conversely, the term \(p_n n\) implies that the
cost of an additional child rises with the quality level \(q\) already
chosen, since each child must meet a higher standard of care or
investment. For instance, a household that chooses to provide more
education or better health care per child incurs an additional burden
when it expands family size, as each child must receive the same
enhanced level of investment. Similarly, a larger family increases the
cumulative cost of quality, even if \(q\) remains fixed, due to the need
to replicate expenditures across more children. In short, the shadow
price of child quality increases with \(n\), and the shadow price of
child quantity increases with \(q\). The cost structure induces a mutual
dependence between quantity and quality, such that any adjustment along
one dimension alters the effective cost of the other.

Becker and Lewis (1973) formalize the mutually reinforcing nature of the
quantity--quality cost structure. An increase in \(n\) raises the total
cost required to sustain a given level of \(q\) for each child, while a
higher level of \(q\) raises the marginal cost associated with having an
additional child. For example, allocating more resources to education or
health per child increases the financial burden of expanding family
size. This interdependence links the two decisions directly. The
household cannot choose \(n\) and \(q\) in isolation; each choice alters
the marginal cost of the other.

Mathematically, the trade-off appears in the first-order conditions of
the household's optimization problem. Let \(\lambda\) represent the
Lagrange multiplier on the full-income constraint.

\[
\mathcal{L} = U(n, q, y) + \lambda \left( I - p_y y - p_n n - p_q n q \right).
\]

The first-order conditions are:

\[
\frac{\partial \mathcal{L}}{\partial n} = U_n - \lambda(p_n + p_q q) = 0, \quad
\frac{\partial \mathcal{L}}{\partial q} = U_q - \lambda p_q n = 0, \quad
\frac{\partial \mathcal{L}}{\partial y} = U_y - \lambda p_y = 0.
\]

Combining the first two yields:

\[
\frac{U_n}{U_q} = \frac{p_n + p_q q}{p_q n}.
\] This condition equates the marginal rate of substitution between
quantity and quality to the ratio of their full marginal costs. The
numerator rises with \(q\), and the denominator rises with \(n\). As one
choice increases, the relative cost of the other becomes higher. This
relationship induces substitution toward the less costly dimension. The
trade-off between quantity and quality arises from the structure of the
budget itself. It does not rely on specific assumptions about utility
curvature or intrinsic substitutability \citep{becker1973interaction}.

This formulation implies two core predictions. Firstly, although both
child quantity and child quality may rise with income, the household's
budget constraint can generate a negative relationship between income
and fertility. As income increases, total spending on children tends to
rise, but the allocation often favors quality over quantity. Becker
illustrated this with the analogy of durable goods: wealthier households
tend to upgrade the quality of a house or a car rather than acquire
additional units. In a similar way, higher-income families often direct
additional resources toward education, nutrition, or health per child.
Within the model, an income increase (\(\mathrm{d}I > 0\)) produces a
direct effect that makes children more affordable and an indirect effect
that discourages fertility. As \(q\) rises, the shadow price of an
additional child also rises. If the marginal utility from higher quality
exceeds that from larger family size, then the substitution effect
outweighs the income effect, leading to a lower optimal \(n\). This
mechanism offers a structural explanation for the demographic
transition: fertility tends to fall as households become richer, even
when preferences remain unchanged.

Furthermore, a similar logic applies to changes in the cost parameters
\(p_q\) and \(p_n\). A decline in \(p_q\), such as through a policy that
lowers the price of education or health care, increases \(q\) and raises
the marginal cost of quantity. This effect reduces optimal fertility. A
rise in \(p_n\), which may reflect higher child-rearing costs or a
greater opportunity cost of parental time, reduces the appeal of larger
families and can shift resources toward child quality. These outcomes
follow from the structure of the budget constraint, without requiring
any explicit preference for quality over quantity. Becker and Lewis
noted that these comparative static results align with observed
patterns. For example, increases in women's wages often reduce fertility
more than they reduce educational spending per child. This asymmetry
reflects the model's central feature: quantity and quality are linked
through their cost structure. An increase in one raises the marginal
cost of the other. The model explains how households make trade-offs
between the number of children and investments in each.

While the early Q--Q models were static (one-period) representations,
subsequent contributions extended the framework to consider fertility
and child investment over multiple periods or even multiple generations.
A central development in this literature was the incorporation of
intergenerational human capital dynamics, where parents derive utility
not only from the number and quality of children in the present, but
also from the long-run outcomes of their offspring. These extensions
allowed child quality to evolve endogenously across time, rather than
being determined solely within a single period.

\bibliographystyle{aea}
\bibliography{references}

\% The appendix command is issued once, prior to all appendices, if any.
\appendix

\section{Mathematical Appendix}


\end{document}
