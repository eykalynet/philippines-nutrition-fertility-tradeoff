% AER-Article.tex for AEA last revised 22 June 2011
\documentclass[2025 Schupf Fellowship Project]{AEA}

% The mathtime package uses a Times font instead of Computer Modern.
% Uncomment the line below if you wish to use the mathtime package:
%\usepackage[cmbold]{mathtime}
% Note that miktex, by default, configures the mathtime package to use commercial fonts
% which you may not have. If you would like to use mathtime but you are seeing error
% messages about missing fonts (mtex.pfb, mtsy.pfb, or rmtmi.pfb) then please see
% the technical support document at http://www.aeaweb.org/templates/technical_support.pdf
% for instructions on fixing this problem.

% Note: you may use either harvard or natbib (but not both) to provide a wider
% variety of citation commands than latex supports natively. See below.

% Uncomment the next line to use the natbib package with bibtex
\usepackage{natbib}

% Uncomment the next line to use the harvard package with bibtex
%\usepackage[abbr]{harvard}

% This command determines the leading (vertical space between lines) in draft mode
% with 1.5 corresponding to "double" spacing.
\draftSpacing{1.5}


% tightlist command for lists without linebreak
\providecommand{\tightlist}{%
  \setlength{\itemsep}{0pt}\setlength{\parskip}{0pt}}


% Add imagehandling




\usepackage{hyperref}

\begin{document}


\title{Nutritional Deficits and the Quantity-Quality Trade-off: Evidence
from an Exogenous Fertility Shock in Low-Income Urban Settings in the
Philippines}
\shortTitle{The Beckerian Quantity-Quality Trade-off in the Philippine
Context}
% \author{Author1 and Author2\thanks{Surname1: affiliation1, address1, email1.
% Surname2: affiliation2, address2, email2. Acknowledgements}}


\author{
  Erika Salvador\\
  Caroline Theoharides,\thanks{
  Salvador: Amherst
College, \href{mailto:esalvador28@amherst.edu}{esalvador28@amherst.edu}.
  Theoharides: Amherst
College, \href{mailto:ctheoharides@amherst.edu}{ctheoharides@amherst.edu}.
  Acknowledgements
}
}

\date{\today}
\pubMonth{05}
\pubYear{2025}
\pubVolume{1}
\pubIssue{1}
\JEL{A10, A11}
\Keywords{first keyword, second keyword}

\begin{abstract}
Abstract goes here
\end{abstract}


\maketitle

\section{Introduction}

The quantity--quality trade-off is one of the most influential ideas in
the economics of the family. This framework, introduced by Becker
\citep{becker1960economic, becker1991allocation}, asserts that parents
who have more children must allocate fewer resources to each child.
Constraints on income and attention lead to reductions in per-child
investments, especially in settings where public provision of services
is weak or inconsistent. The trade-off helps explain long-run trends in
fertility decline, human capital accumulation, and intergenerational
poverty in low- and middle-income countries (LMICs)
\citep{rosenzweig1980testing, schwarze2003does}.

Despite this theoretical foundation, most empirical tests of the
trade-off have prioritized educational outcomes as the primary measure
of child ``quality.'' Studies often rely on proxies such as school
enrollment, standardized test scores, or years of completed schooling
\citep{black2005more, angrist2010effects}. These indicators indeed
encapsulate important aspects of human capital formation, yet they
obscure other equally vital dimensions. Among these, nutrition stands
out as both foundational and predictive. Adequate nutrition supports
cognitive development and raises productivity later in life
\citep{hoddinott2013adult}. Children who experience chronic
undernutrition face biological constraints that limit their ability to
benefit from schooling, regardless of enrollment status or household
income \citep{grantham2007development}. Analyses that exclude
nutritional outcomes therefore risk understating the full scope of the
quantity--quality trade-off.

This paper addresses this gap by studying how an exogenous increase in
fertility affected child nutrition in the Philippines. In 1990, the
mayor of Manila issued an executive order that prohibited modern
contraceptives in the city's public health system. Health centers could
no longer distribute birth control pills, condoms, or IUDs. Local
officials also removed family planning materials and instructed
providers to offer only natural methods. Because the national government
did not impose similar restrictions, this policy created a natural
experiment. Poor families in Manila experienced a sudden reduction in
access to contraceptive services, while other urban households in
comparable regions retained access \citep{dumas2019fertility}.

I use this policy discontinuity to examine the relationship between
fertility and child nutrition. The analysis draws on multiple waves of
the Philippine Demographic and Health Surveys (DHS), which offer
nationally representative data on household structure, maternal
characteristics, and child anthropometrics. I focus on stunting (low
height-for-age) and wasting (low weight-for-height) as outcome
variables. These indicators capture long-term and short-term nutritional
stress, respectively, and predict later-life productivity, disease risk,
and mortality \citep{victora2008maternal}.

The empirical strategy proceeds in three steps. First, I replicate prior
work to confirm that the contraceptive ban increased fertility among
affected women \citep{dumas2019fertility}. Second, I estimate
reduced-form effects of exposure to the policy on child nutrition.
Third, I examine heterogeneity by maternal education, household wealth,
and access to prenatal care. Families with fewer resources may face
tighter constraints when household size increases. If the
quantity--quality trade-off holds in the nutritional domain, then
children in these households should face a higher risk of malnutrition.

This study contributes to the literature in several ways. It provides
rare causal evidence that links fertility shocks to nutrition rather
than education. It expands the definition of child quality to reflect
physiological outcomes. It also highlights the long-run consequences of
local reproductive policy in a middle-income democracy. In the wider
policy landscape, the Philippine case offers a warning. Governments that
restrict reproductive autonomy may unintentionally weaken child health
and human capital formation. As countries pursue goals related to
nutrition, health equity, and gender rights, evidence from natural
experiments such as this one can inform the design of more inclusive and
sustainable population policies.

Sample figure:

\begin{figure}
Figure here.

\caption{Caption for figure below.}
\begin{figurenotes}
Figure notes without optional leadin.
\end{figurenotes}
\begin{figurenotes}[Source]
Figure notes with optional leadin (Source, in this case).
\end{figurenotes}
\end{figure}

Sample table:

\begin{table}
\caption{Caption for table above.}

\begin{tabular}{lll}
& Heading 1 & Heading 2 \\
Row 1 & 1 & 2 \\
Row 2 & 3 & 4%
\end{tabular}
\begin{tablenotes}
Table notes environment without optional leadin.
\end{tablenotes}
\begin{tablenotes}[Source]
Table notes environment with optional leadin (Source, in this case).
\end{tablenotes}
\end{table}

References here (manual or bibTeX). If you are using bibTeX, add your
bib file name in place of BibFile in the bibliography command. \% Remove
or comment out the next two lines if you are not using bibtex.

\bibliographystyle{aea}
\bibliography{references}

\% The appendix command is issued once, prior to all appendices, if any.
\appendix

\section{Mathematical Appendix}


\end{document}
