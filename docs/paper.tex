% AER-Article.tex for AEA last revised 22 June 2011
\documentclass[]{AEA}

% The mathtime package uses a Times font instead of Computer Modern.
% Uncomment the line below if you wish to use the mathtime package:
%\usepackage[cmbold]{mathtime}
% Note that miktex, by default, configures the mathtime package to use commercial fonts
% which you may not have. If you would like to use mathtime but you are seeing error
% messages about missing fonts (mtex.pfb, mtsy.pfb, or rmtmi.pfb) then please see
% the technical support document at http://www.aeaweb.org/templates/technical_support.pdf
% for instructions on fixing this problem.

% Note: you may use either harvard or natbib (but not both) to provide a wider
% variety of citation commands than latex supports natively. See below.

% Uncomment the next line to use the natbib package with bibtex
\usepackage{natbib}

% Uncomment the next line to use the harvard package with bibtex
%\usepackage[abbr]{harvard}

% This command determines the leading (vertical space between lines) in draft mode
% with 1.5 corresponding to "double" spacing.
\draftSpacing{1.5}


% tightlist command for lists without linebreak
\providecommand{\tightlist}{%
  \setlength{\itemsep}{0pt}\setlength{\parskip}{0pt}}


% Add imagehandling




\usepackage{hyperref}

\begin{document}


\title{Nutritional Deficits and the Quantity-Quality Trade-off: Evidence
from an Exogenous Fertility Shock in Low-Income Urban Settings in the
Philippines}
\shortTitle{The Q-Q Trade-off in the Philippine Context}
% \author{Author1 and Author2\thanks{Surname1: affiliation1, address1, email1.
% Surname2: affiliation2, address2, email2. Acknowledgements}}


\author{
  Erika Salvador\\
  Caroline Theoharides\thanks{
  Salvador: Amherst
College, \href{mailto:esalvador28@amherst.edu}{esalvador28@amherst.edu}.
  Theoharides: Amherst
College, \href{mailto:ctheoharides@amherst.edu}{ctheoharides@amherst.edu}.
  This research was made possible by the Schupf Fellowship at Amherst
  College. I am indebted to my faculty adviser, Professor Caroline
  Theoharides, for her mentorship and support throughout the development
  of this project. I also wish to thank Faculty Director Professor
  Amelie Hastie and the campus partners whose institutional support
  enabled this work. I am further grateful to the Departments of
  Economics and Mathematics \& Statistics for their academic support.
  The views expressed and any errors contained herein are entirely my
  own responsibility.
}
}

\date{\today}
\pubMonth{06}
\pubYear{2025}
\pubVolume{}
\pubIssue{}
\JEL{J13, I15, O15}
\Keywords{fertility shocks, child nutrition, quantity-quality
trade-off, urban poverty, Philippines}

\begin{abstract}
This paper examines whether increased fertility affects early-life
nutritional outcomes in low-income urban households. I exploit a natural
experiment created by a 1990 policy in Manila, Philippines, which banned
modern contraceptives from city-run health facilities. Using a
difference-in-differences framework and nationally representative data
from the Philippine Demographic and Health Surveys, I estimate the
reduced-form impact of the policy on child height-for-age and
weight-for-height. {[}Will add more after data analysis{]}
\end{abstract}


\maketitle

\section{Introduction}

The trade-off between child quantity and child quality is a foundational
concept in the economics of the family. First articulated by
\citet{becker1960economic} and extended in subsequent models of
household behavior \citep{beckerlewis1973, beckertomes1976}, this
framework posits that parents allocate finite resources---both financial
and non-financial---across children. An increase in fertility reduces
the resources available per child and, under binding constraints, may
lead to lower investments in health, education, and other forms of human
capital. This mechanism has served as an explanatory model for changes
in fertility behavior and the evolution of population structures in low-
and middle-income countries.

Empirical investigations of the quantity--quality trade-off have focused
primarily on educational outcomes. Studies in both high-income and
low-income settings have examined the effects of fertility on school
enrollment, grade progression, test scores, and completed years of
schooling
\citep{rosenzweig1980testing, black2005more, angrist2010effects}. These
outcomes serve as accessible proxies for long-run human capital
accumulation, but they represent only one dimension of child quality.
Other outcomes, such as nutritional status, are early-onset or
biologically constrained. They also tend to be less responsive to
remediation later in life. A child who suffers from chronic malnutrition
may exhibit permanently reduced cognitive capacity and face limits in
physical development that affect long-run productivity regardless of
subsequent educational access
\citep{hoddinott2013adult, grantham2007development}.

The exclusion of nutritional outcomes from much of the empirical
literature leaves an important dimension of the trade-off untested.
Nutritional investments in early childhood are essential to early
childhood development and long-term outcomes
\citep{victora2008maternal, hoddinott2013adult}. They shape brain
development, immune system functioning, and physical stature, and they
have been shown to predict later-life earnings and health outcomes
across a wide range of settings
\citep{grantham2007development, alderman2006long}. The biological
irreversibility of early-life nutritional deficits further distinguishes
them from other forms of investment. Educational deficits may be
partially remediable; nutritional failures often are not. A credible
estimate of the trade-off between fertility and child quality must
account for nutrition if it aims to assess the full set of consequences
associated with fertility shocks.

This study addresses this gap by examining the nutritional effects of a
localized, exogenous increase in fertility in the Philippines. In 1990,
the mayor of Manila implemented an executive policy that prohibited the
provision of modern contraceptives in all city-run health facilities.
The order removed access to pills, condoms, intrauterine devices, and
related public health materials and instructed healthcare providers to
offer only natural family planning methods. This policy remained in
place for nearly a decade and affected only the jurisdiction of the
Manila city government. The national government did not implement a
comparable restriction, and surrounding cities within Metro Manila
continued to provide access to modern contraceptives. The policy thus
created a spatial and temporal discontinuity in contraceptive access
that was uncorrelated with underlying fertility preferences or
concurrent shifts in household income or governance. As a result, the
Manila ban serves as a quasi-experimental source of variation in
fertility exposure among poor urban households.

I use this natural experiment to estimate the causal effect of increased
fertility on child nutrition. The analysis relies on nationally
representative data from multiple waves of the Philippine Demographic
and Health Survey (DHS), which provide data on household structure and
maternal characteristics, as well as measurements of child
anthropometry. The outcomes of interest are height-for-age and
weight-for-height z-scores, which serve as standardized indicators of
chronic and acute malnutrition, respectively. These outcomes are widely
used in the global health and development literature and capture
nutritional deprivation over both long and short time horizons
\citep{victora2008maternal}. The empirical strategy follows a
difference-in-differences design that compares child outcomes in Manila
and comparable urban areas before and after the onset of the policy.

The identification strategy rests on two key assumptions. First, in the
absence of the contraceptive ban, nutritional trends in Manila would
have evolved in parallel with those in comparison cities. Second, any
other policy or economic shocks affecting Manila during the study period
must not coincide precisely with the timing and scope of the
contraceptive policy. I test these assumptions using falsification
checks, placebo comparisons, and robustness specifications that include
city-specific time trends, maternal fixed effects, and controls for
baseline demographic differences.

The analysis proceeds in three stages. I first replicate existing work
\citep{dumas2019sex} to confirm that the contraceptive ban led to an
increase in fertility among affected women. I then estimate reduced-form
effects of policy exposure on nutritional outcomes for children under
five years of age. Finally, I examine heterogeneity in effects across
subsamples defined by maternal education, household wealth, and access
to prenatal care. These dimensions serve as proxies for household
resource availability and capacity to buffer the nutritional
consequences of fertility increases.

This study contributes to the literature in several important ways. It
provides new evidence on how increases in fertility---caused by policy
restrictions on family planning---can affect child nutrition in poor,
urban communities. Most past research has focused on education, but this
study expands the idea of child quality to include biological measures
such as stunting and wasting. It also adds to the small number of
studies that use unexpected changes in reproductive health policy to
examine long-term effects on children's well-being. More broadly, the
results show that local restrictions on family planning can
unintentionally harm children's health, especially in settings where
families already face poverty, food insecurity, and limited public
services.

\section{Review of Related Literature}

\subsection{Theoretical Background}

Gary Becker's early work reframed fertility as an economic decision and
drew on microeconomic theory to how families choose whether and how many
children to have. In his 1960 paper, Becker challenged the then-dominant
view that fertility declines with income were simply the result of
better access to contraception or changing social values. Instead, he
proposed that childbearing decisions could be modeled using the tools of
consumer choice theory. Within this framework, children are treated as
goods that provide utility, with households allocating resources to
balance their desire for more children (quantity) against the desire to
invest in each child (quality).

Becker argued that children serve both consumption and production
purposes. They bring direct satisfaction to parents and may also
contribute economically, especially in agrarian or informal settings. In
the Philippines, many children aged 5--17 work informally---through
street vending or helping in family businesses---to support household
income, with an estimated 1.09 million children working in 2023
\citep{manilabulletin2024}. This economic role aligns with Becker's
framework and may partly explain why some low-income households continue
to prefer larger families. At the same time, fertility tends to decline
as income rises and per-child investments increase. According to the
2017 National Demographic and Health Survey, women in the poorest wealth
quintile had an average of 4.5 children, compared to just 2.0 in the
richest \citep{psa2018}. This inverse relationship between income and
fertility, observed globally, also appears in the Philippine context:
poorer households often maintain larger families, while wealthier ones
prioritize child quality.

\citet{beckerlewis1973} build on these conceptual foundations to
introduce the formal concept of a quantity--quality (Q--Q) trade-off.
They develop a utility-maximizing framework in which parents allocate
resources between the number of children and the quality of investments
in each child. The model formalizes how parents maximize utility from
consumption \(c\), number of children \(n\), and child quality \(q\):

\[
U = U(c, n, q)
\]

subject to the full income constraint: \[ I = c+n(p_n + q \times p_q) \]
where \(p_n\) denotes the direct cost associated with bearing each
additional child (e.g., delivery, basic needs), and \(p_q\) represents
the marginal cost of investing in quality per child, such as health care
or education. The trade-off is reflected in the structure of the cost
function, which reveals interaction effects: the marginal cost of
improving child quality rises as the number of children increases, and
similarly, the cost of an additional child rises with the level of
quality investment per child. This complementarity is captured through
the positive cross-derivatives:

\[
\frac{\partial MC_q}{\partial n} > 0 \quad \text{and} \quad \frac{\partial MC_n}{\partial q} > 0
\]

In short, the more children a family has, the more difficult it becomes
to allocate sufficient resources---such as time, nutrition, and
education---to each child.

While the static Q--Q model explains the immediate trade-offs families
face between the number and quality of children, it remains limited in
scope by assuming a single-period decision framework, which fails to
account for intertemporal linkages---for instance, how early investments
in child health or education may constrain or enhance later household
choices---and overlooks how forward-looking behavior, such as
anticipated returns to child quality in the form of future earnings or
social mobility, may influence present-day household choices. To address
this, \citet{becker1976child} developed a dynamic model that explicitly
incorporates intergenerational mobility. In this model, parents account
for how their fertility and investment choices shape the human capital
outcomes of their children---for example, how a decision to space births
more closely can constrain parental resources and reduce per-child
investments in health and education. The resulting dynastic utility
function captures both present and future-oriented preferences:

\[
U = u(c) + \beta \cdot n \cdot V(q)
\]

where \(\beta\) denotes child altruism, or the extent to which parents
value their children's future well-being relative to their own
consumption. A higher \(\beta\) indicates stronger intergenerational
concern, meaning parents are more willing to sacrifice current resources
to improve their children's prospects. For example, a low-income family
in the Philippines might choose to send a child to private school
despite financial constraints to increase the child's future earnings or
social mobility. Conversely, a lower \(\beta\) implies a stronger
emphasis on present needs and diminished willingness to invest in
long-term child outcomes. This may occur in households facing severe
income insecurity, where sending children to work may be prioritized
over education. On the other hand, \(V(q)\) represents the expected
return on child quality. A high \(V(q)\) implies that parents expect
substantial future benefits from quality investments, such as access to
better-paying jobs or improved life expectanc, making such investments
more worthwhile than having additional children. A low \(V(q)\), in
contrast, suggests that even with significant investments, the perceived
or actual gains are limited, often due to barriers like poor school
quality, weak labor demand, or geographic constraints.

To formalize the trade-off similarly as in the static model, we again
impose the full income constraint. In optimizing this problem, the
first-order condition for child quality becomes:

\[
\beta \cdot V'(q) = p_q + n \cdot \frac{\partial p_q}{\partial q}
\]

This condition implies that the marginal benefit of child quality,
scaled by altruism \(\beta\), must equal the marginal cost of improving
that quality. As family size \(n\) increases, the cost of maintaining or
increasing quality also rises. This reinforces the quantity--quality
trade-off: families with more children may face sharper constraints in
deepening investments per child. In contexts like the Philippines, where
income constraints are binding and child quality has high private
returns, this model predicts lower fertility and increased education
spending among upwardly mobile households.

\bibliographystyle{aea}
\bibliography{references}

\% The appendix command is issued once, prior to all appendices, if any.
\appendix

\section{Mathematical Appendix}


\end{document}
